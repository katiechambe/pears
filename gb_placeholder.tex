


Low mass galaxy pair fractions are understudied across cosmic time. Needed is a self-consistent framework to select both low and high mass galaxy pairs in order to connect observed pair fractions to cosmological merger rates across all mass scales and redshifts. Here, we construct a sample of low mass ($\rm 10^8<M_*<5\times10^9\,\Msun$) and high mass ($\rm 5\times10^9<M_*<10^{11}\,\Msun$) subhalo pairs from the TNG100 simulation from $z=0-4.2$. Pairs belong to the same FoF group, meaning they are isolated, physically associated, and not contaminated by projection effects. Stellar masses are assigned using abundance matching. We find that the low mass pair fraction increases from $z=0-2.5$, while the high mass pair fraction peaks at $z=0$ and is constant or slightly decreasing at $z>1$. At z=0 the low mass major pair fraction is four times lower than that of high mass pairs, consistent with findings for cosmological merger rates. Our results indicate that pair fractions can faithfully reproduce trends in merger rates if galaxy pairs are selected appropriately. Specifically, static pair separation limits applied equivalently to all galaxy pairs do not recover the  evolution of low and high mass pair fractions. Instead, we advocate for separation limits that vary with the mass and redshift of the system, such as separation limits scaled by the  estimated virial radius of the primary ($r_{sep}< 1 R_{vir}$). Finally, we place isolated mass-analogs of Local Group galaxy pairs (MW-M31, MW-LMC, M31-M33, LMC-SMC) in a cosmological context, quantifying for the first time the frequency of finding isolated analogs of such pairs as a function of redshift. 


In particular galaxy pairs like the MW-M31 system are 



However, applying a separation cut that is a function of mass and redshift recovers the difference in the behavior of low and high mass pairs at all redshifts. 

  
  There has been no systematic study to ensure that any differences in \lcdm\ simulations can be captured in observational surveys. 
  In this study we utilize the TNG100 simulation to select isolated low mass and high mass pairs self-consistently to quantify pair fractions from $z=0-4.2$.
  We find that the pair fraction of high mass pairs peaks at $z=0$ and is constant or slightly decreasing with increasing redshift for $z>1$, while low mass pair fractions increase from $z=0$ to $z\sim2.5$ before leveling out from $z=2.5-4.2$.
  We also assess the impact of separation-based pair selection criteria on the resulting pair fractions, and find that physical separation cuts applied equivalently to all pairs eliminates the ability to distinguish the behavior of low and high mass pair fractions across all redshifts. 
  However, applying a separation cut that is a function of mass and redshift recovers the difference in the behavior of low and high mass pairs at all redshifts. 
  Our analysis is consistent with previous findings of a redshift and mass dependence of the merger fraction and merger rate of galaxy pairs in simulations. 
  We argue that in order for observational pair fraction campaigns to recover the true pair fractions, and thus the true merger rates, of galaxy pairs, separation criteria must evolve as a function of redshift and mass of each pair.
  Finally, we study the frequency of isolated mass-analogs of galaxy pairs in the Local Group in a cosmological context, and quantify for the first time the frequency of finding such isolated pairs as a function of redshift and separation.