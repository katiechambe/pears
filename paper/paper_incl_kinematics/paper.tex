\documentclass[twocolumn]{aastex631}

% Packages
\usepackage{microtype}  % ALWAYS!
\usepackage{amsmath}
\usepackage{amsfonts}
\usepackage{amssymb}
\usepackage{multirow}
\usepackage{tikz}
\usepackage{xcolor}
\usepackage{soul}

\definecolor{pink}{RGB}{232,132,161}
\definecolor{yellow}{RGB}{255,213,0}

\newcommand{\kc}[1]{\textcolor{yellow}{\textbf{kc: #1}} }
\newcommand\shadetext[2][]{%
  \setbox0=\hbox{{\special{pdf:literal 7 Tr }#2}}%
  \tikz[baseline=0]\path [#1] \pgfextra{\rlap{\copy0}} (0,-\dp0) rectangle (\wd0,\ht0);% 
  }
\newcommand{\gb}[1]{\shadetext[left color=blue, right color=red, middle color=lime, shading angle=45]{\textbf{g: #1}} }
% \newcommand{\ecite}[1]{\textcolor{pink}{\textbf{: #1}} }
% \newcommand{\e}[1]{\textcolor{yellow}{\textbf{: #1}} }

\newcommand{\remove}[1]{\textcolor{red}{#1}}
\newcommand{\add}[1]{\textcolor{green}{#1}}

\newcommand{\mlg}{\ensuremath{M_{\rm LG}}}
\newcommand{\mmto}{\ensuremath{M_{\rm M31}}}
\newcommand{\mmw}{\ensuremath{M_{\rm MW}}}
\newcommand{\vtan}{\ensuremath{v_\textrm{tan}}}
\newcommand{\vrad}{\ensuremath{v_\textrm{rad}}}
\newcommand{\ms}[1]{\ensuremath{M_{*{#1}}}}
\newcommand{\mud}{\ensuremath{\mu_\delta}}
\newcommand{\mua}{\ensuremath{\mu_\alpha^*}}
\newcommand{\bov}{\ensuremath{\boldsymbol{v}}}
\newcommand{\boldx}{\ensuremath{\boldsymbol{x}}}
\newcommand{\vtrav}{\ensuremath{\bov_{\rm travel}}}
\newcommand{\xtrav}{\ensuremath{\boldx_{\rm travel}}}
\newcommand{\pos}[2]{\ensuremath{\boldx_{\rm #1 \to #2}}}
\newcommand{\vel}[2]{\ensuremath{\bov_{\rm #1 \to #2}}}
\newcommand{\mwbary}{\ensuremath{\textrm{MW}_\textrm{bary}}}
\newcommand{\mwouter}{\ensuremath{\textrm{MW}_\textrm{halo}}}
\newcommand{\mwdisk}{\ensuremath{\textrm{MW}_\textrm{disk}}}
\newcommand{\reflabel}[1]{\ensuremath{^{\mbox{\scriptsize{#1}}}}}
\newcommand{\scsep}{\ensuremath{\rm r_{sep}/r_{vir}}}
\newcommand{\scvel}{\ensuremath{\rm v_{rel}/v_{vir}}}

% Style tweaks
% \renewcommand{\twocolumngrid}{\onecolumngrid}
% \setlength{\parindent}{1.1\baselineskip}
% \sloppy\sloppypar\raggedbottom\frenchspacing

%%%%%%%%%%%%%%%%%%%%%%%%%%%%%%%%%%%%%%%%%%%%%%%%%%%%%%%%%%%%%%%%%%%%%%%%%%%%%%%%
\shorttitle{Frequency of galaxy pairs}
\shortauthors{Chamberlain et al.}

%%%%%%%%%%%%%%%%%%%%%%%%%%%%%%%%%%%%%%%%%%%%%%%%%%%%%%%%%%%%%%%%%%%%%%%%%%%%%%%%
\graphicspath{{./}{../plots/paper1/}}
\input{math_definitions.tex}

% Affiliations
\newcommand{\affuofa}{University of Arizona, 933 N. Cherry Ave,
    Tucson, AZ 85721, USA}

%% This is the end of the preamble.  Indicate the beginning of the
%% manuscript itself with \begin{document}.

\begin{document}

\title{
  Frequency of low and high mass galaxy pairs in the IllustrisTNG cosmological simulation
}

\author[0000-0001-8765-8670]{Katie~Chamberlain}
\affiliation{\affuofa}

\author[0000-0003-0715-2173]{Gurtina Besla}
\affiliation{\affuofa}


\begin{abstract}
  
  \todo{Name data sets and refer to their names }
\end{abstract}

%%%%%%%%%%%%%%%%%%%%%%%%%%%%%%%%%%%
\section{Introduction} \label{sec:intro}

\begin{itemize}
    \item Galaxy mergers are a feature of LCDM and support the growth of galaxies across cosmic time.
    \item The frequency of finding pairs tells us about how often we might expect: starbursts, SMBH mergers, the importance/role of mergers in building stellar mass/morphologies of galaxies today, SFHs, etc. 
    \item pair fraction at high masses has been well studied both obs and theoretically.
    \item in the era of jwst, these studies will be pushed to lower masses and high z, however, from a theory persp[ective, pair fractions at low masses have been under-studied (first 2 ish paragraphs ~ )
    \item in this study, we'll be quantifying the pair fraction of low mass pairs, and self consistently comparing to the pair fraction of high mass systems
    \item There are reasons to believe that high and low mass (Mstar = 1e8-5e9Msun) pairs might evolve differently, so it's important to study the low mass end too. 
    \item Also, major and minor mergers are important for different reasons, so studying them as separate systems is well motivated.
        - is major high mass vs major low mass different, etc since these two types of mass ratios lead to different things 
        - importance of mergers in high vs low mass galaxies (major mergers may be less important to SFH of Universe etc - look at sabrina's papers)
    \item Systematically understanding the underlying difference between low and high mass pairs, Projected vs. physical 3D separations/pairs Two routes of studying galaxy pairs are through observations and simulations, but simulations allow us to fully understand the 3D position and velocity of objects. 
    \item There are studies of high mass (Mstar = 5e9-1e11Msun) pairs. These studies are typically done in projection for comparison to observational campaigns.
    \item There are studies of low mass pairs, but only at low z, or as satellites of more massive systems (analogs of LG). 
    \item In this work, we aim to characterize the pair fraction of both high and low mass galaxies, major and minor pairs, consistently as a function of time. 
\end{itemize}
cite Stierwalt for low mass pairs 4485/4490 (iso vs non iso pairs) 
Pearson Besla 2016 
Pearson Besla 2018 4485/4490
Luber Pearson Besla 
Nevin 

Discussion section: 
what does this mean for our understanding of the LG (isolated mass analogs) 
ngc 4490/4485 as another low z example (minor dwarf pair) 


%%%%%%%%%%%%%%%%%%%%%%%%%%%%%%%%%%%
%%%%%%%%%%%%%%%%%%%%%%%%%%%%%%%%%%%
 \section{Methodology}\label{sec:methods}
We aim to quantify and characterize the frequency (or occurrence rate) and bulk kinematic properties of galaxy pairs in cosmological simulations as a function of cosmic time. 
In particular, we study dwarf galaxy pairs and massive galaxy pairs that are of similar stellar and halo mass to a variety of galaxy pairs in the Local Group. 
We categorize these pairs into one of four pair types: massive major pairs, such as the MW+M31; massive minor pairs, such as M31+M33 or MW+LMC; dwarf major pairs, such as M33+LMC; and finally, dwarf minor pairs, such as the LMC+SMC.
\kc{can maybe move these two sentences to a discussion section}

To this end, we utilize the IllustrisTNG simulations to select halo pairs at each snapshot according to a strict set of selection criteria as outlined in the remainder of this section. 
In Sec.~\ref{sec:methods-sims}, we provide motivation for and details of the simulation utilized in this study.
Sec.~\ref{sec:methods-halos} outlines the initial mass cuts used to define our full sample of halos, from which we will select pairs.
Sec.~\ref{sec:methods-am} outlines the abundance matching prescription we used to associate dark matter subhalos with stellar masses.
Sec.~\ref{sec:methods-pairs} describes the second set of selection criteria we apply to obtain the full sample of pairs.
Finally, Sec.~\ref{sec:pairprops} details some of the properties of our entire pair sample, including the total number of pairs and stellar mass ratios.


%%%%%%%%%%%%%%%%%%%%%%%
% Sim details section %
    \subsection{Simulation details} \label{sec:methods-sims}
        % Brief description of TNG hydro and which run we use.
    The \tng\ project ~\citep{TNG1, TNG2, TNG3, TNG4, TNG5} consists of a suite of dark-matter-only $N$-body as well as full physics cosmological simulations consistent with the \textit{Planck2015} \lcdm\  cosmology~\citep{Planck2015}.
    In this study we utilize data from the main high resolution, full physics run of the (110.7\Mpc)$^3$ volume simulation, \textsl{TNG100-1}, which includes dark matter and baryons and follows the evolution of $1820^3$ dark matter particles in 100 snapshots from $z=127$ to $z=0$. 

      % subfind, sublink, and groups and subhalos
    We utilize the group catalogs produced by the \texttt{SUBFIND} algorithm~\citep{springel2001,dolag09}. 
    These catalogs consist of halos, defined using the Friends-of-Friends (FoF) algorithm~\citep{davis1985} to link nearby particles, and their associated subhalos, which are over-dense and gravitationally bound dark matter structures compared to the background. (From here, we will refer to FoF halos as "groups" and their subhalos as )
      % We also use merger catalogs generated by XX for YY purpose. 
    We also use merger trees provided by the \texttt{SUBLINK} algorithm~\citep{rg15}, which tracks subhalos between snapshots using their DM particle data, in order to trace subhalo mass growth and merger histories.
    
    The large volume of this simulation offers a large statistical sample of halos to study in a cosmological context, but is also high enough resolution ($m_{\rm DM} = 7.5\times10^6\Msun$, $m_{\rm gas} = 1.4\times10^6\Msun$) that dwarf halos ($\sim1\times10^{9-10.5}\Msun$) halos are resolved into roughly hundreds or more particles, well above the resolution limit of 32 DM particles for halos identified with \texttt{SUBFIND} \kc{fix this} 20 gas and star particles used to construct baryonic merger trees.  

        %  this MIGHT go in a different section? am I talking about this because I want to show why we use hydro instead of dark? 
    Baryonic process are known to affect the structure and formation of dark matter halos of low mass ($M_*\lesssim 1\times10^9\Msun$) subhalos in simulations~\citep[see e.g.][and references therein]{Sales:2022}.
    Additionally, the shallow potential well of dwarf galaxies compared to their massive counterparts may lend themselves more easily to disruption from baryonic processes such as supernova feedback and reionization~\citep{}, resulting in fewer low mass halos ($\lesssim 5\times10^{10}$) in hydrodynamic simulations~\citep{vogelsberger14B}.
    % Thus, the number of primaries  we expect that our choice of simulation will result in 
% End sim details     %
%%%%%%%%%%%%%%%%%%%%%%%

%%%%%%%%%%%%%%%%%%%%%%%%%%    
% Choosing halos section %
    \subsection{Choosing dwarf and massive halos in TNG} \label{sec:methods-halos}
    % start off with group selection 
        % brief explanation of FoF groups in sim
    Each snapshot from the TNG100 group catalogs consist of FoF groups, large halos of closely associated DM particles, and the subhalos they contain, which are smaller bound structures of DM and the gas and star particles nearest to them. 
    We will refer to these two structures as "groups" and either "halos" or "subhalos" respectively, from here on. 
    Galaxies form in these subhalos, so our pair sample will consist of pairs of subhalos within these larger group structures.

    \kc{be more prescriptive. }
    We are interested in the inherent dynamics and frequency of pairs of halos in the field.
    Thus, we want our pairs to consist of some of the most massive halos in each group, and ensure that the pair is not dynamically influenced by nearby massive groups. 
    By selecting pairs from the most massive halos of a group, a rough isolation criteria is inherent in our selection due to the nature of the FoF algorithm, which groups together nearby particles. 
    To confirm that our sample , we have confirmed that nearly 100\% of the most massive subhalos in each group are outside of the Hill radius of any nearby group or subhalo with greater mass.

    The first selection criteria that we invoke is a cut on the total group mass, $M_{G}$ (i.e., a mass cut on the FoF group mass, given by \texttt{Group\_M\_TopHat200} in the \tng\ group catalogs). 
    We define the group mass range for dwarf groups as $8\times 10^{10} < M_{G,d} < 5\times 10^{11}\Msun$
    %, which will host mass analogs of the LMC+SMC and LMC+M33 pairs. 
    Massive groups range from $1\times10^{12} < M_{G,m} < 6.5\times10^{12}$, which at the lower mass end will host MW+LMC pairs, and MW+M31 pairs at the higher mass end. 
    The group mass criteria is fixed for all redshifts, so at high z, the dwarf pairs we select may be the progenitors of higher mass systems at $z=0$.
    Note that all masses used in this analysis are physical units, such that masses from the group catalogs are divided by $h$ to get units of \Msun. 

    Selecting dwarf pairs from the dwarf group mass range also ensures that we are not selecting satellite systems in the halos of more massive systems. 
    The LMC+SMC, for example, are highly perturbed by the influence of the MW, and thus we do not pick dwarf pairs from high mass groups that may contain large nearby MW-mass analogs.
    
        % explain what halos we will move forward with. 
    For each group that passes the group mass cut, we create a catalog of all subhalos that pass a ``current mass" criteria: the subhalo mass at the given snapshot (same snapshot) halo mass must be $M_h > 1\times10^{9}\Msun$ (the \texttt{SubhaloMass} field in the group catalogs).\footnote{Note that for the TNG100 hydro run, the halo mass is the total sum of all bound dark matter \textit{and} baryonic particles.}
    This lower mass threshold ensures that our halos are resolved into $>100$ particles, and thus should be robust to the SUBFIND and SUBLINK algorithms even at the lowest mass end.  
    
    We then utilize the merger tree catalogs to determine the maximum halo mass ever achieved by each halo. 
    The set of all subhalos at each snapshot that pass the group mass and current mass cuts constitutes our entire subhalo sample. 
    From here, we will use further selection criteria to narrow our sample down to pairs.
    \kc{at this stage, this "Isolated Group" catalog contains XX }

% End choosing halos     %
%%%%%%%%%%%%%%%%%%%%%%%%%%    

%%%%%%%%%%%%%%%%%%%%%%%%%%%%%%    
% Abundance matching section %
    \subsection{Abundance matching} \label{sec:methods-am}
        % what is abundance matching
    Abundance matching is a technique used to associate dark matter halo mass with specific galaxy properties.
    This allows us to associate dark matter halos from the \tng\ simulation with observationally motivated galaxy properties. % maybe observationally constrained, instead? 
    In particular, we will be utilizing a stellar mass to halo mass (SMHM) relationship to assign stellar masses to each of the subhalos in our sample. 

    % few details about the msoter relationship. how was it determined?
        % why are we using abundance matching 
    The \tng100-1 subhalo catalogs do already contain a stellar mass field as computed in the simulation by simulating star formation and  AGN feedback, stellar feedback, galactic winds, etc.    
    However, there are a couple reasons we opt to assign stellar masses to our halos, rather than use those from the simulation. 
    
    Primarily, applying an abundance matching prescription to assign stellar masses to each of the dark matter halos allows us to account for the observed spread in the stellar mass-halo mass relationship. 
    This is particularly important in the low mass regime ($M_h \lesssim 10^{11}\Msun$) where the spread in the SMHM function is large. 
    In addition, using stellar masses as calculated from abundance matching allows us to side step any simulation-dependent stellar mass effects.
    In the case of \tng100 specifically, it is known that the stellar masses of low mass galaxies are often overestimated compared to the galaxy luminosity function at the low mass end\cite{}. 
    Using an observationally-calibrated abundance matching prescription will thus lead to a more robust result for comparison with and predictions for observations. 
    Finally, utilizing an abundance matching prescription allows us to directly compare results between dark matter only and hydro simulations, since the stellar masses are assigned the same way. 
    We make brief note of results for our equivalent analysis using the dark matter only TNG100-1-Dark simulation as well in Sec.~\ref{sec:results}.
    
        % what am prescription are we using? 
    We use the abundance matching relationship presented in~\citet{Moster2013}, which provides an analytic prescription to assign stellar masses to dark matter halos. 
    The~\citet{Moster2013} relationship is a function of the halo mass, evolves with redshift, and includes terms to account for the systematic scatter in the SMHM relationship, with larger scatter at lower halo masses.
    Additionally, stellar mass is defined separately for central halos (``hosts") and for infalling satellites (``subhalos"), where the first is a function of the virial mass of the halo and its redshift, $M_{vir}$ and $z$, and the second is a function of the mass and redshift at the time of infall, $M_{inf}$ and $z_{inf}$. 
    This assumes that the satellite halos are at the peak of their stellar mass before infall, and neglects the affect of star formation or stellar stripping within the host halo. 

        % what do we use for the mass ~ 
    In this work, we will use the peak halo mass \Mpeak\ from the \sublink\ merger trees to calculate subhalo stellar masses. 
    \citet{Munshi2021} found that the stellar mass of subhalos at $z=0$ in the ``Marvel-ous Dwarfs" and ``DC Justice League" zoom simulations are more closely correlated with the peak halo mass ($M_{peak}$) than the $z=0$ halo mass for halos with peak halo mass $10^8<M_{\rm peak}<10^{11} \Msun$. 
    % separate citation for peak halo mass and stellar mass relatiosnhip for M > 1e11Msun 
    Using the peak halo mass additionally allows us to remain robust to scenarios in which a secondary has formed most of its stars, then loses a significant portion of its dark matter content through tidal interactions with a primary, but retains the bulk of its stellar content.
        
        % regarding the redshift dependence ~
    We additionally assume that the stellar mass of both the primary and secondary halo can change as a function of redshift after the secondary has entered the primary's halo and do not immediately cease star formation. 
    This assumption is consistent with findings from~\cite{Akins2021}, which found that massive dwarf satellites ($M_*\approx 10^8-10^9\, \Msun$) entering MW-mass host halos are rarely quenched, and with~\cite{geha13} which finds that dwarfs $>1\Mpc$ from a MW-type galaxy are often star forming and rarely quenched.
    Additionally, the SAGA survey has found that large satellites of MW-type galaxies are often very blue, with infall into the halo spurring high rates of star formation due to the large gas fraction in dwarfs~\citep{saga}. 
    Thus, we calculate stellar masses for our subhalos using $M_{\rm peak}$ and the redshift of the current snapshot to calculate the stellar mass of each subhalo, and to account for the additional stellar mass that can form after infall. 

    As mentioned in a footnote in Sec.~\ref{sec:methods-halos}, the total halo mass of subhalos in the hydro simulation includes the mass of both dark matter and baryonic particles. 
    In TNG100-1, the DM particle mass is decreased XX\% compared to the DM particle mass in \tng100-1 Dark to account for the additional mass of baryons, such that the total mass of the halo will remain roughly the same between the dark matter only and hydro simulations. 
    Since the~\citet{Moster2013} AM relation was calibrated using DM subhalos from the \texttt{MilleniumII} dark matter only simulation, we will use the total subhalo mass from the TNG100-1 hydro simulation as a proxy for the corresponding DM halo mass in the TNG100-1 Dark run, \texttt{SubhaloMass}. 

        % realizations? 
    For each subhalo at each snapshot in our full subhalo sample (see Sec.~\ref{sec:methods-halos}), we create a "realization" of the catalog for , 1000 stellar mass values from the SMHM relationship to account for the systematic uncertainty in subhalo stellar masses.
    Each realization is treated as an independent sample of galaxy stellar masses, which allows us to report realistic spreads of pair properties.
    Thus, we create 1000 separate pair catalogs, and compute pair properties (pair frequency, separations, etc) for each individual abundance matching realization.
    \kc{what is the name of this catalog? (Isolated Group cats + 1000 Stellar Mass) }

% End Abundance matching %
%%%%%%%%%%%%%%%%%%%%%%%%%%   

%%%%%%%%%%%%%%%%%%%%%%%%%%    
% Pair selection section %
    \subsection{Pair selection}\label{sec:methods-pairs}
    The full pair sample is constructed from the set of all subhalos and their abundance matching realizations. Each pair will consist of a ``primary" more massive object, and a ``secondary" less massive companion. 
    \kc{We are starting with XX catalog, we'll sprinkle on some selection criteria, then store YY and ZZ. }

    % justify why we want to use stellar mass to define things
    %A majority of the literature uses the definition of "major" and "minor" pairs based on the stellar mass of the objects. This is true as well for observational pair samples, such as Lotz 2011, wherein the halo mass of the objects is unknown, and thus the stellar mass is used to define the mass ratio of the pair. In order to reproduce statistics that can be meaningfully compared to observations/can be used to inform observations, we opt to define our pair sample via the stellar masses from the abundance matching prescription given in Sec.~\ref{sec:methods-am}. 

    %%% 
    % begin subsubsection
    \subsubsection{Selecting primaries}
        Primary galaxies (equivalently, ``primary subhalos") are the most massive galaxy of their group by stellar mass. Thus, each group can have at most one primary galaxy.  
        
        We apply mass cuts on the stellar mass of the primary to define our sample of dwarf primaries and massive primaries. Each stellar mass realization is treated independently, such that the primary of a group may change between stellar mass realizations. Additionally, some groups may not have a ``primary" in one realization, but will in others. \kc{}
    
        For each stellar mass realization, we rank order the subhalos of each group by stellar mass, then perform a mass cut specific to the type of group: dwarf primaries from dwarf mass groups, and massive primaries from massive mass groups (see Sec.~\ref{sec:methods-halos} for group mass criteria). 
        We define the stellar mass range for dwarf primaries and massive primaries as:
        \begin{align*} 
        \mbox{\textbf{dwarf primaries:}}&\, 1\times 10^{8}< \rm M_{*} < 5\times10^{9}\Msun \\ 
        \mbox{\textbf{massive primaries:}}&\, 5\times 10^{9}< \rm M_{*} < 1\times 10^{11}\Msun.
        \end{align*}
        \kc{justification for this stellar mass range? LG relation to put into context these ranges }
        The primary stellar mass range is chosen to replicate a set of pairs with masses similar to LMC--SMC and MW--M31. \kc{remove "pairs"}
        
        
        The stellar mass criteria is fixed, and does not change as a function of redshift. 
        The median number of primaries in a single stellar mass realization at $z=0$ is XX dwarf primaries and XX massive primaries (see Sec.~\ref{sec:pairprops} for more info). 
        
        The abundance matching prescription has a spread of $\sim 1$dex at $M_{h}\sim 10^{10}\Msun$~\citep{Moster2013},
        \kc{move to AM section}%thus the set of selected primaries in each realization is typically unique. 
    
        % explain special cases that people might wonder about.
        %A subhalo will not be selected as a primary if it is not the first most massive halo in the group, i.e., there are no subhalos in the group that are more massive than the primary (by stellar mass). However, s
        Since our selection is based on stellar mass, and there is a large spread in the SMHM relation, the primary for any realization of the catalog may not be the most massive subhalo in terms of total subhalo mass. 
        %This is by design, since our selection criteria are meant to more closely replicate observational pair selection criteria which use the stellar mass ratio between the primary and companion to define a major or minor pair, rather than the DM halo mass. 


    % end  primary subsubsection
    %%% 
    
    %%% 
    % begin secondary subsubsection
    \subsubsection{Selecting secondary companions}
        The second most massive galaxy by stellar mass in each group is then used to define the secondary companion for each primary. 
        As before, the selection of secondary companions occurs independently for each stellar mass realization, and for each snapshot, such that the population of secondaries in each realization will have a different stellar mass distribution.
        %we collect 1000 different sets of pairs for each redshift.

    \subsubsection{To create the pair catalog,}
        We create two categories of pairs: 
        major pairs and minor pairs, such that the secondary companion galaxies are selected via the stellar mass ratio between the secondary and primary.     
        % Since we have already rank ordered the halos in each stellar mass realization by their stellar mass, the stellar mass ratio will never exceed 1.   
        The stellar mass ratio criteria for major and minor pairs are:
        \begin{align*} 
        \mbox{\textbf{major pair:}}&\, \frac{1}{4}\leq \frac{\rm M_{*2}}{\rm M_{*1}}< 1 \\ 
        \mbox{\textbf{minor pair:}}&\, \frac{1}{10}\leq \frac{\rm M_{*2}}{\rm M_{*1}} < \frac{1}{4}
        \end{align*}
        \kc{give the reason for this look at z0.png}

        \kc{For each pair we store the pair separation in physical units and relative velocity. We require that each pair have a minimum separation of $15 \kpc$ between the primary and secondary, which is XX times the smoothing length of the simulation. }
        \kc{need to explain how the separations and velocities are computed from the catalog etc. }
        At smaller separations, more and more mass will be attributed to the central subhalo, which can lead to confusion in the SUBFIND catalogs. %the merger tree histories. \kc{the }.
      
        If a primary subhalo does not have a companion that meets the stellar mass ratio criteria, separation criteria, or lower subhalo mass criteria (see Sec.~\ref{sec:methods-halos}), the subhalo will still be considered a primary. 
        We refer to all primaries without companions as ``isolated primaries."
        \kc{move this down to the secondary section}
        If a subhalo passes the stellar mass cut for a primary, but does not have an associated companion from the selection criteria for a secondary (see next subsection), the subhalo will still be considered a primary. We refer to all primaries without companions as ``isolated primaries."
        The total set of primaries includes both paired primaries and isolated primaries.
    
        The pair catalog contains blah blah blah. 
        1000 sets of pairs at each snapshot blah blah 

        The pair sample consists of 1000 sets of pairs at each snapshot, each set representing a different realization of the abundance matching prescription. 
        At each snapshot, we calculate the median occurrence rate and kinematic values for each of the 1000 realizations. 
        Each of the following plots shows the parameter's median and percentile spread calculated from the set of 1000 medians from each realization.
        
        how do we use these catalogs? Interested in the aggregate properties of the full sample, and so we will treat blah blah 
    
    % end secondary subsubsection
    %%%     
% Pair selection end %
%%%%%%%%%%%%%%%%%%%%%%

% Methods end %
%%%%%%%%%%%%%%%
%%%%%%%%%%%%%%%

%%%%%%%%%%%%%%%%%%%%%%%%%
% begin pair prop plots %  
\begin{figure*}[htb]
    \centering
    \includegraphics[width=\textwidth]{counts_1000.png}
    \caption{(Top) Median number of dwarf (left) and massive (right) primaries and pairs as a function of redshift.
    Shaded areas show the 1-99 percentile range of the median from 1000 abundance matching realizations, as described in Sec.~\ref{sec:results}.
    There are approximately 8 times as many dwarf primaries as massive primaries. 
    % dwarf galaxies
    The dwarf primary count (solid) is at a minimum at $z=4$, rises to a maximum by $z\sim1$, then decreases from $z=1\to0$. 
    The dwarf pair count (dashed) peaks at $z\sim2$.
    % Massive galaxies (right)
    The massive primary count (pink solid line in top right panel) behaves similarly to the dwarf primary count, with a minimum count at $z=4$ which rises to a maximum of $\sim2000$ halos per realization at $z\sim1$, then decreases slightly by $\sim5\%$ to $\sim$1800 halos by $z=0$. 
    Unlike dwarf pairs whose pair count peaks at $z=2$, pair count for massive galaxies (pink dashed line) follows similar behavior to the total massive primary count, with a minimum at $z=4$, and increasing to $z=\sim 1$ before leveling off. At very low redshift, the pair count and primary count have opposite behavior, with the pair count \textit{increasing} at very low redshifts of $z=0-0.25$ and \textit{peaking} at $z=0$.
    \kc{where does it peak and what is }
    (Bottom) Fraction of pairs per primary as a function of redshift (i.e., the ratio between dotted line and solid line in each column). Dwarf and massive pair fractions are roughly flat for $z=2.5-4$, but display opposite behavior for $z=0-2.5$. The dwarf pair fraction decreases steadily from $\sim0.24$ to $\sim0.08$, a decrease of roughly $65\%$, while the massive pair fraction is roughly flat between $z=1-2.5$ with an average of $0.31$, before spiking sharply from $\sim 0.29$ to $\sim 0.36$ between $z=0-0.25$, an increase of $25\%$.}
    \label{fig:counts}
\end{figure*}

\begin{figure*}[htb]
    \centering
    \includegraphics[width=\textwidth]{smrdist_1000.png}
    \caption{\kc{using all 1000 realizations - combined sample of XX catalog} Stellar mass ratio distribution of all dwarf (top) and massive (bottom) pairs. Major pairs (solid lines) are defined as pairs with mass ratio $\ms{2}/\ms{1} > 1/4$, while minor pairs (dotted lines) are defined as pairs with stellar mass ratio $1/10<\ms{2}/\ms{1}<1/4$. Overall, the stellar mass ratio distribution of major and minor pairs of dwarf and massive galaxies show little evolution from $z=4$ (right) to $z=0$ (left). 
    Major pairs make up $51-55\%$ of the full sample of pairs at every redshift for both dwarf and massive pairs.}
    \label{fig:massratio}
\end{figure*}
% end pair prop plots %  
%%%%%%%%%%%%%%%%%%%%%%%

%%%%%%%%%%%%%%%%%%%%
% Pair prop. start %
\section{Sample: Overview of pair properties} \label{sec:pairprops}
    Utilizing the pair catalog, We compute the # of prims in each realization of the 
    The full pair sample contains 1000 individual sets of pairs at each snapshot, for all 100 snapshots of the \tng100-1 simulation. 
   


    Fig.~\ref{fig:counts} shows the redshift evolution of the median number of dwarf and massive primaries and pairs we identify over the redshift range $z=0-4$.   
    For each snapshot, we calculate the median number of primaries and of pairs in each realization within the Pair Catalog, yielding a set of 1000 medians. 
    The solid line and dashed lines are the median of this set of 1000 medians, and the shaded regions show the 1-99\% spread of the set.
    
    The number of identified primaries is smallest at $z=4$, and rises to a maximum around $z=1$ for both dwarf and massive primaries.        
    There are approximately 8 times as many dwarf primaries as massive primaries. 
    
    The median dwarf primary count (green solid line in top left panel) reaches a maximum of $\sim15000$ halos at $z\sim0.6$, then decreases by $\sim16\%$ to 12000 halos at $z=0$. 
    The dwarf pair count (green dashed line) peaks much earlier at $z\sim2$ with $\sim3000$ pairs, and decreases to $\sim1000$ pairs at $z=0$.
    From $z=4$ to $z=2$, the number of dwarf primaries doubles, as does the number of pairs due to the growth of structures over cosmic time.
    The massive primary count (pink solid line in top right panel) reaches a maximum median of $\sim2000$ halos (in one realization) at $z\sim1$, then remains approximately constant to $z=0$. 
    The pair count for massive galaxies (pink dashed line) behaves similarly to the primary count, increasing from $z=4$ to $z\sim1$, then leveling off to $z=0$. 

    

    The bottom panel of Fig.~\ref{fig:counts} shows the number of pairs divided by the number of primaries, or the fraction of pairs per primary as a function of redshift. 
    Dwarf and massive pair fractions are roughly flat for $z=2.5-4$, but display opposite behavior for $z=0-2.5$. The dwarf pair fraction decreases steadily from $\sim0.24$ to $\sim0.08$, a decrease of roughly $65\%$, while the massive pair fraction is roughly flat between $z=1-2.5$ with an average of $0.31$, before spiking sharply from $\sim 0.29$ to $\sim 0.36$ between $z=0-0.25$, an increase of $25\%$.

    include all 1000 realizations as a single "Combined Sample"
    We show the distribution of stellar mass ratios between the secondary and primary of each dwarf and massive pair in Fig.~\ref{fig:massratio} from the Pair Catalog, com 
    \kc{explain how we compute this}
    all realizations.
    % Major pairs make up $51-55\%$ of the full sample of pairs at every redshift for both dwarf and massive pairs.
    In general, the shape of the distribution remains constant from $z=0-4$ for both dwarfs and massive pairs, and between the dwarf and massive pairs at each redshift. 
    Thus, the typical stellar mass ratio of pairs in our sample is independent of mass scale and of redshift. \kc{how much more likely are you to find 1/4 than 1:1 things? what about 1:10 to 1:4?}
            
% Pair prop. end %
%%%%%%%%%%%%%%%%%%


  
% \begin{table*}[htb]
%   \centering
%     \begin{tabular}{lcc}
%      & Dwarf Pairs & Massive Pairs \\\hline\hline
%     % Group Mass & $8\times10^{10} < \rm M_g < 5\times10^{11} M_\odot$ & $1\times10^{12} < \rm M_g < 4\times10^{12} M_\odot$  \\
%     % Subhalo Max Mass &  &  \\
%     % Subhalo Current Mass & $>1\times10^{10} \Msun$ & $> 5\times10^{11} \Msun$ \\
%     Primary &$1\times10^{8} < \msam < 5\times10^{9} \Msun$ & $5\times10^{9} < \msam < 1\times10^{11} \Msun$\\\hline
%     \end{tabular}
%     \caption{\label{table:mass}Selection criteria for dwarf and massive pairs.}
%     \end{table*}

% equivalent snapshot table
% \begin{table*}[tbp]
%   \centering
%   \begin{tabular}{l|cc|cccc} % increase this to 7 columns
%     \hline \hline
%    & {Snapshot} & {Number of primaries} & {Number of pairs}\\
%    & \ill & \tng  & \illd & \illh & \tngd & \tngh \\ 
%   \hline
%   z = 0   &   135  &   99   & $11431.5_{-24.42}^{68.38}$ & $12650.0_{-25.84}^{46.84}$ & $13255.0_{-32.60}^{56.72}$ & $12035.0_{-69.24}^{20.36}$\\
%   z = 1   &   85   &   50   &  $14769.0_{-30.24}^{47.28}$ & $16568.0_{-32.20}^{69.60}$ & $17445.5_{-40.86}^{51.90}$ & $15946.5_{-59.38}^{39.22}$ \\
%   z = 2   &   68   &   33   &  $12602.5_{-31.98}^{45.94}$ & $14174.0_{-12.48}^{62.32}$ & $15413.5_{-34.62}^{28.86}$ & $14334.0_{-29.44}^{45.24}$ \\
%   z = 3   &   60   &   25   &     $8533.5_{-54.50}^{39.46}$ & $9390.0_{-63.40}^{41.32}$ & $10901.0_{-44.64}^{46.56}$ & $10106.5_{-47.90}^{39.74}$  \\
%   % z = 3.7   &   54   &   21   \\
%   \hline \hline
%   \end{tabular}
%   \caption{\label{tab:equiv-snapshot} The snapshot number of the original \ill\ and the \tng\ simulations at redshifts $z=0-3$, as well as the median number dwarf (massive) primaries that were selected with $1\times 10^{8} < \msam < 5\times 10^{9} \rm M_\odot$ ($5\times 10^{9} < \msam < 1\times 10^{11} \rm M_\odot$) in XX abundance matching realizations. \kc{fill in with numbers after doing more realizations} \kc{include takeaway}}
%   \end{table*}


%%%%%%%%%%%%%%%%%%%%%%%%%
%%%%%%%%%%%%%%%%%%%%%%%%%
% Begin Results section %

\section{Results: The frequency and kinematics of dwarf and massive pairs}
We have created catalogs of isolated low and high mass galaxy pairs from $z=0-4$ in the TNG100-1 simulation. 
In this section, we will analyze the occurrence rate and kinematic properties (physical separations and relative velocities) for major and minor pair types across cosmic time. 

In Sec.~\ref{sec:results-frac}, we show the redshift evolution of the fraction of primaries with a major or minor companion (the ``pair fraction") and compare the results for dwarf pairs and massive pairs.
In Sec.~\ref{sec:results-kinematics}, we look at the redshift evolution of the separations and relative velocities between primary and secondary halos of each pair. % need to specify this is snapshot by snapshot?
In Sec.~\ref{sec:results-scaled}, we compare the separations and velocities scaled by the FoF group's virial properties. 
Finally, in Sec.~\ref{sec:results-virialcut}, we take subsets of the Pair Catalog, selecting only pairs within some fraction of the virial radius, and illustrate the impact on the recovered low and high mass pair fractions. 

%%%%%%%%%%%%%%%%%%%%%   
% Definitions start %
% \subsection{Definitions} \label{sec:results-defs}
        
% Definitions end %
%%%%%%%%%%%%%%%%%%% 


%%%%%%%%%%%%%%%%%%%%%%%%%
% Pair fraction section %
    \subsection{Pair fraction}\label{sec:results-frac}
    We determine the occurrence rate of a pair type by calculating the pair fraction. 
    The pair fraction is defined here as the ratio of the number of pairs to the total number of primaries (isolated + paired):
    $$\rm \frac{N_{pair-type}}{N_{primaries}}.$$
    For example, the dwarf major pair fraction is calculated as the number of dwarf major pairs divided by the number of dwarf primaries. 
    The pair fraction of dwarf or massive pairs with a major or minor companion gives the likelihood of finding a companion of a certain mass type about a primary identified in the field.
    
    Fig.~\ref{fig:pairratio} shows the pair fractions calculated for massive and dwarf pairs (including both major and minor pairs separately) as a function of time for $z=0-4$. 
        % differences between massive and dwarf pairs? 
    The dwarf pair fraction evolves distinctly from that of massive pairs.
    Dwarf pair fractions (both major and minor) start at their minima at $z=0$, and increase by about 200\% by $z\sim2-2.5$, at which point the pair fractions level off and remain constant from $z=3-4$. 
    On the other hand, massive pair fractions reach their maxima at $z=0$, then abruptly declining to $z\sim0.25$, before remaining approximately constant from $z=1-4$. 
    
        % difference panels
    The bottom panels of Fig.~\ref{fig:pairratio} shows the dwarf pair fraction subtracted from the massive pair fraction (labelled ``$\rm M-D$"). 
    The pair fraction difference reaches a maximum at $z=0$ for both major and minor pairs, and declines to $z\sim2-2.5$ at which point the difference levels out close to 0, where major dwarf and massive pairs, as well as minor dwarf and massive, are equivalently common. 
    
        % differences between major and minor pairs? 
    The median pair fraction of minor pairs is less than that of major pairs of the same mass range at all redshifts. 
    This is due to the lower stellar mass ratio limit we put on our pairs. 
    Expanding the mass ratio criteria of our minor pair sample to include stellar mass ratios between 1/10-1/100 in our minor pair sample increases the fraction of minor pairs by a factor of $>2$ for massive pairs and between $~1.5-2x$ for dwarf pairs. \kc{are there any additional takeaways from this? }
    
        % what do we learn from this? what are the takeaways? where do we discuss this in more detail?
    Overall, these results show that massive and dwarf pair counts evolve differently over time, particularly at very low redshift, despite the pair fractions being roughly equal at higher z. 
    These differences could be a result of the different formation mechanisms at play in the evolution of the systems at these two different mass scales, or alternatively, from differences in satellite dynamics between dwarf and massive systems (see Sec.~\ref{sec:results-kinematics}). 
    The implications for the difference in the evolution of pair fractions for dwarf and massive pairs across time is discussed in detail in Sec.~\ref{sec:discussion}.
    
        % pair fraction plot
    \label{sec:results}
    \begin{figure*}[htp]
      \centering
      \includegraphics[width=\textwidth]{pairfrac_1000.png}
      \caption{
        % Top plots ~
        (Top) Median pair fraction, defined as the fraction of dwarf or massive primaries with a major (solid) or minor (dashed) companion by stellar mass (see Sec.~\ref{sec:methods-pairs}). 
        Shaded areas show the 1-99 percentile range of the median from 1000 abundance matching realizations. 
          % massive pairs
        The massive pair fraction (pink) remains approximately constant for $z>1$, with an average pair fraction of 0.16-0.17 for major pairs and 0.14-0.15 for minor pairs. Between $z=0.25$ and $z=0$, the massive major pair fraction increases to 0.207, while the minor pair fraction increases to 0.152.
          % dwarf pairs
        Dwarf pair fractions, shown in green, are approximately constant from $z=3-4$, then decline monotonically to $z=0$. The major pair fraction is $0.041$ at $z=0$ and $0.126$ at $z=3$, while the minor pair fractions are $0.034$ and $0.111$ at $z=0$ and $z=3$, respectively. 
        % Bottom plots ~
        (Bottom) The median subtracted (massive - dwarf) pair fraction difference, with the shaded $1$-$99$ percentile range from 1000 abundance matching realizations. The pair fraction difference increases with decreasing redshift, peaking at $z=0$ with a difference of 0.166 for major pairs and 0.111 for minor pairs. This shows that the redshift evolution of the pair fractions of massive pairs and dwarf pairs proceed differently, particularly at low redshift.}
      \label{fig:pairratio}
    \end{figure*}
% end pair fraction 
%%%%%%%%%%%%%%%%%%%%%%%%%

%%%%%%%%%%%%%%%%%%%%%%%%%%%%%
% beging Kinematics results %
    \subsection{Kinematic evolution}\label{sec:results-kinematics}
    We also analyze the kinematic behavior of our pair sample as a function of time. 
    Instantaneous pair separations are defined as the relative separation between the primary and secondary of each pair in the snapshot, and are computed in physical $\kpc$, and \textit{not} in comoving $\kpc/h$. 
    Similarly, relative velocities are defined as the relative velocity difference in the same snapshot between the primary and secondary of each pair, and are computed in physical $\kms$.
    
    In this section, we compute the median relative separations and velocities of all pairs in each snapshot, each of which includes the set of pairs from 1000 stellar mass realizations.
    We separate the population of dwarf and massive pairs further into major and minor pairs to show how the stellar mass ratio between the primary and secondary may affect the kinematics. 
    
    % separations %
        \subsubsection{Separations}
        %% text on separation vs. redshift plot
        %% intro to separations & bulk results
        % We calculate the median relative separations between the primary and secondary halos of the pairs identified in our dwarf pair sample and massive pair sample. 
        % We split the results by the stellar mass ratio of each pair, dividing the sample into "major" pairs and minor pairs, which allows us to compare the behavior of different mass ratio systems individually in the dwarf and massive mass range.
        Our results are 
        illustrated in Fig.~\ref{fig:sep}, which shows the redshift evolution of the median relative separations for dwarf pairs and massive pairs. 
        %% the behavior of massive and dwarf pairs
        We show that the median relative separations between the primary and secondary of both dwarf and massive pairs increases with decreasing redshift, regardless of pair type. 
        A combination of physical mechanisms is likely spurring the increase in median pair separations at low $z$. 
        For example, high $z$, low separation pairs may undergo mergers at earlier times, removing them from the sample at later snapshots, thus increasing the median separation over time. 
        Alternatively, mass accretion, from the cosmic web and/or from mergers, causes dark matter halos to become both more massive and more extended as $z\to0$. 
        Thus, the growth of structure that is expected in $\Lambda$CDM could naturally lead to an increase in typical pair separations at low $z$.
        We explore the relationship between halo growth and relative pair separations in Sec.~\ref{sec:results-scaled}.
        
        %% the DIFFERENCE between dwarf and massive pairs
        Dwarf and massive pairs exhibit similar relative separation trends as a function of time, however, the scale of the typical relative separations of pairs in our sample is set by the mass halo of the primary.
        Dwarf pairs and massive pairs have a relative separation difference of $\sim150\kpc$ at $z=0$, and a difference of $\sim40$ at $z=4$.   
        This is expected, as more massive halos have larger spheres of influence and more substructure. 
        \kc{maybe move this sentence to discussion:}But, the difference between dwarf and massive pair relative separation ranges, and their evolution over time begs the question: what does this mean for observational pair selection criteria at different mass scales and at different redshifts? 
        We leave the discussion of the implications our relative separation results on pair selection criteria for Sec.~\ref{sec:discussion}. 
        
        %% the difference between major and minor pairs
        In addition, the bottom panels of Fig.~\ref{fig:sep} shows the median subtracted difference between the relative separations calculated for major pairs and minor pairs of both mass types. 
        There is a small offset in the median separations of major and minor dwarf pairs (between $10-25\,\kpc$ from  $z=0-1$ and approximately $5\,\kpc$ from $z=1-4$), and a larger offset for major and minor massive pairs (between $25-100\,\kpc$ from  $z=0-1$ and approximately $10-40\,\kpc$ from $z=1-4$). 
        The major-minor offset is likely a result of more major pairs merging more quickly (thus at earlier times and higher $z$) due to the increased effect of dynamical friction, thus leading to a population of low relative separation minor pairs that have yet to merge.  
    
        %% separation plots
        \begin{figure*}[htp]
          \centering
          \includegraphics[width=\textwidth]{sep_1000.png}
          \caption{(Top) Median physical separation in kpc between the primary and secondary halos of dwarf pairs (left) and massive pairs (right) as a function of redshift. 
          Shaded areas show the 1-99\% percentile range of the median from 1000 abundance matching realizations. 
          Major pairs are shown in solid lines, and minor pairs are depicted with dashed lines.
          In general, for both dwarf pairs and massive pairs, the average physical separation of the two halos in a pair increases for as the redshift decreases from $z=4\to0$. 
          % Dwarf pairs
          The typical separation between major (minor) dwarf pairs increases from $42(38)\,\kpc$ at $z=4$ to a peak of $128(109)\,\kpc$ at $z=0$.
          % Massive Pairs
          The separation between major (minor) massive pairs likewise increases from $81(71)\,\kpc$ at $z=4$ to $288(205)\,\kpc$ at  $z=0$.
          % Difference plots
          (Bottom) The median difference and 1-99\% range (as above) between the median separation of major pairs and minor pairs.
          At dwarf mass scales, minor companions tend to be $\sim10-20\kpc$ closer to their primary than major companions, while minor massive pairs have a relative separation that is $10-50\kpc$ less than major pairs.
            }
          \label{fig:sep}
        \end{figure*}
    
    % velocities %
        \subsubsection{Relative Velocities}
        %% text on velocity vs. redshift plot
        %% intro to velocity & bulk results
        % In addition to the separations between the primary and secondary halos of our pair sample, we also calculate the median relative velocity between the halos of each pair.
        % Once again we split the results into "major" and "minor" pairs for both the dwarf and massive pair sample to get a better understanding of the impact of the mass of the secondary. 
        In Fig.~\ref{fig:vel}, we show the median relative velocity between the primary and secondary halos of dwarf and massive pairs as a function of redshift. 
        %% the behavior of massive and dwarf pairs
        We find that the median relative velocity of a pair decreases towards lower redshift, regardless of pair type.
        The median velocity of a major dwarf pair decreases linearly from $130$ at $z=4$ to $70\,\kms$ at $z=0$. Likewise, the median velocity for major massive pairs decreases from $260$ at $z=4$ to $170\,\kms$ at $z=0$, though with more spread at high $z$.
        \todo{} As the circular velocity of a secondary object about the primary is inversely proportional to the radius, it is not surprising to find a decrease in the relative velocity for pairs at lower redshift due to the decrease in concentration/increase in the virial radius of the halo as $z\to0$. \kc{citation needed}
        See Sec.~\ref{sec:results-scaled} for more details.
        
        %% the DIFFERENCE between dwarf and massive pairs
        As in the case of relative separations of pairs, dwarf and massive pairs show a similar evolutionary trend in relative velocity as a function of redshift, though at different scales determined by the mass of the primary.
        Massive pairs have a median relative velocity $\sim100\kms$ larger than dwarf pairs at $z=0$, and $\sim140\kms$ at $z=4$. \kc{see comment}
        %This is also not particularly surprising, since the circular velocity prop to $( M/r)^1/2,$, and massive primaries may be >10x as massive as dwarf primaries, while the separation between a massive primary and it's secondary will only be 2x further than a dwarf pair.
    
        %% the difference between major and minor pairs
        Fig.~\ref{fig:vel} also includes a panel showing the subtracted median difference at each snapshot of the minor pair from the major pair. Unlike the relative separation, the relative velocities do not have a strong dependence on mass ratio of the pairs. 
        The offsets between the relative velocities of dwarf major and minor pairs is between $0-5\,\kms$, and between $0-25\,\kms$ for massive major and minor pairs. \kc{see comment} %does it make sense that minor pairs are faster?
        
        %% velocity plots 
        \begin{figure*}[htp]
          \centering
          \includegraphics[width=\textwidth]{vel_1000.png}
          \caption{
            (Top) Median relative velocity (in \kms) between the primary and secondary halos of dwarf pairs (left) and massive pairs (right) as a function of redshift. 
            Shaded areas show the 1-99 percentile range of the median from 1000 abundance matching realizations. 
            The overall behavior of the redshift evolution of pair relative velocities is nearly identical for dwarf and massive pairs.
            As the redshift decreases from $z=4$ to $z=0$, the average relative velocity of a pair decreases for both dwarf and massive pairs.
            % brief describe
            The median relative velocity of a dwarf pair (major or minor) is $70\,\kms$ at $z=0$ and $133\,\kms$ at $z=4$, and of a massive pair is $170\,\kms$ at $z=0$ and $260\,\kms$ at $z=4$ ($295\,\kms$ for a minor pair at $z=4$). 
            % Difference plot
            (Bottom) The median relative velocity difference between major and minor pairs ("Maj - Min"), and 1-99\% shaded region.
            There is little difference between the relative velocity of major and minor dwarf pairs (bottom left), with minor pairs having a relative velocity of $0-5\,\kms$ higher than major pairs. 
            Massive pair types (bottom right) also have little velocity difference at low redshift, although minor pairs have a relative velocity between $1-20\,\kms$ faster than major pairs at higher redshift ($z>\sim 2.5$).
            }
          \label{fig:vel}
        \end{figure*}
% end kinematic results %        
%%%%%%%%%%%%%%%%%%%%%%%%%

%%%%%%%%%%%%%%%%%%%%%%%%
% begin scaled results %
    \subsection{Scaled Kinematics}\label{sec:results-scaled}
    %% intro to scaled kin. evolution & bulk results
    In the previous section, Sec.~\ref{sec:results-kinematics}, we show that the separations and relative velocities of dwarf and massive pairs evolve similarly as a function of redshift. 
    However, the magnitude of pair separations and velocities are set by the mass scale of the pair. 
    For example, a dwarf pair at $z=0$ has a median relative velocity of $70\kms$, while a massive pair will have a median velocity of $175\kms$ at $z=0$.
    
    To explore the impact of halo growth and mass scale the kinematics pairs, we study the relative separation of a primary and secondary halo scaled by the virial radius ($r_{vir}$) of the FoF group, and the relative velocity ($v_{vir}$) divided by the circular speed of the FoF group at the virial radius. 
    That is, 
    $$\rm r_{scaled}=\scsep$$
    $$\rm v_{scaled}=\scvel$$
    $$\rm v_{vir}=\sqrt{G M_{vir}/r_{vir}},$$
    where $\rm r_{vir}$ is the group radius \texttt{Group\_R\_TopHat200} in physical $\kpc$ from the group catalogs, and $\rm M_{vir}$ the total group mass \texttt{Group\_M\_TopHat200} in $\Msun$.
    
    In Fig.~\ref{fig:scaled}, we show the median scaled separation and scaled velocity for dwarf and massive major pairs as a function of time. 
    Minor pairs show the same trends, and so we have omitted them here for brevity.
    
    %% the behavior of massive and dwarf pairs
    Overall, the median scaled separation of dwarf pairs and massive pairs do not evolve significantly differently from $z=4\to0$.
    In fact, the median separation of halos in a pair is smaller than the group virial radius at $z<2.5$, but is larger than the virial radius at $z>2.5$.
    %maybe not bound at high z, though could become bound by low z by mass accretion ? 
    Dwarf pairs tend to have slightly larger separations relative to the group virial radius than massive pairs at almost all redshifts, though the spread of median scaled separations is large for massive pairs at $z>2$. 
    
    The median scaled velocities of dwarf pairs and massive pairs increase at lower redshift (see right panel of Fig.~\ref{fig:scaled}). In fact, dwarf and massive pairs evolve almost identically for $z=0-4$, and thus, the redshift evolution of a pair at either of these mass scales is independent of the total mass of the system. 
    We also find that pairs at $z<0.5$ have relative velocities in excess of their virial velocity, but are smaller than the virial velocity for $z=0.5-4$.
     
    %% distributions of scaled kinematics
    In Fig.~\ref{fig:scaled-dist}, we show the distribution of scaled separations and velocities at redshifts $z=0,1,2,3,4$. 
    The distributions show roughly equivalent shape and extent for both dwarf and massive scaled kinematics, and do not vary significantly with redshift. 

    % behavior of massive vs. dwarf 
    Overall, our results imply that there is little difference between high mass and low mass pairs in the redshift evolution of pair velocities, though pair separations seem to have a weak dependence at low redshift on the type of pair. Inner regions of the halo , high mass pairs tend to be even closer, likely due to 
    This appears opposite to the physical separations, but when you scale to the total mass of the system, the 
    Fof groups are increasng in size . \todo{}
    
    %% scaled evolution plot
    \begin{figure*}[htp]
      \centering
      \includegraphics[width=\textwidth]{scaledcombo_1000.png}
      \caption{ \label{fig:scaled}
          %  Left
          (Left) Median scaled separation as a function of redshift (see Sec.~\ref{sec:methods} for the definitions of scaled separation and velocity). All shaded regions show $X-XX\%$ percentile spread from 1000 abundance matching realizations. 
          % any overall trends?
          At $z>2$, a majority of all pair types have separations greater than the group virial radius, while at $z<2$, a majority of pairs have smaller separations.  
          Massive major pairs at low redshift tend to have slight smaller scaled separations than dwarfs at $z<\sim2$, but vary significantly at higher redshifts. 
          % Right 
          (Right) Median scaled velocity as a function of redshift.
          % any overall trends?
          All pairs show a peak in scaled velocity around $z=0$. From $z=0-0.5$, a majority of both dwarf and massive pairs have relative velocities larger than the circular velocity of their group at the virial radius. At $z>0.5$, however, the majority of pairs have $v_{rel}<v_{vir}$. The scaled velocity evolves nearly identically for dwarf and massive pairs.
          % Bottom 
          (Bottom) The scaled separation/velocity of dwarf major pairs subtracted from that of massive major pairs, indicated by ``M-D".
          The scaled separation difference shows small differences between the dwarf and massive population, with dwarf pairs having slightly higher scaled separation for $z<1$. The scaled velocity difference is $\sim0$ at all redshifts.
          Minor pairs (not shown here) follow the same trends as major pairs in both the massive and dwarf regime.}
    \end{figure*}
    
    %% scaled distribution plot
    \begin{figure*}[htp]
      \centering
      \includegraphics[width=\textwidth]{scaledcombodist_1000.png}
      \caption{
          Scaled separation distribution (left) and scaled velocity distribution (right) at $z=0,1,2,3$, and $4$, normalized by the total number of pairs at the corresponding redshift. 
          % the point? what am I trying to get across
          The first thing to note is that there is relatively little redshift-dependent change in the overall shape and extent of each distribution, and thus the scaled quantities are only loosely independent of redshift. 
          Additionally, the shape and extent of the dwarf pair (top) and the massive pair (bottom) distributions are nearly identical, thus the scaled quantities are also independent of the halo mass scale. Thus, the kinematics of dwarf pairs and massive pairs evolve roughly equivalently when normalizing by the halo mass.
          Unscaled versions of this plot are discussed in Sec.~\ref{sec:discussion}.
      }
      \label{fig:scaled-dist}
    \end{figure*} 
% end scaled results %
%%%%%%%%%%%%%%%%%%%%%%
    
% end kinematics %
%%%%%%%%%%%%%%%%%%

\subsection{Separation cuts as fxn of virial radius}
put in plots~ 
% end Results section%
%%%%%%%%%%%%%%%%%%%%%%
%%%%%%%%%%%%%%%%%%%%%%


%%%%%%%%%%%%%%%%%%%%%%%%%%%%
% Begin Discussion section %
\section{Discussion}\label{sec:discussion}
\subsection{Implications for the differences in pair fractions of dwarf and massive pairs}
In Sec.~\ref{sec:results-frac}, we presented the pair fraction of massive and dwarf galaxy pairs and found that the relative frequency of the two populations evolve distinctly from $z=0-4$. 
While the frequency of massive pairs compared to the number of massive primaries is roughly constant and become more common at low redshift, dwarf major and minor pairs become more than 50\% less common at $z=0$ compared to $z=2$. 
Since scaled separations and velocities do not change drastically between dwarf and massive pairs, the difference in the evolution of the pair fractions indicates that the merger processes of each population may proceed differently.

% One scenario that could explain this behavior is a very rapid merger timescale for dwarf pairs, and a slower merger timescale for massive pairs. 
% In this case, dwarf pairs will merge quickly after formation, thus depleting the number of dwarf pairs at later times. 
% Simultaneously, the merged remnant of a dwarf pair is likely to join the population of isolated primaries until star formation finally pushes the stellar mass up into the massive primary regime.
% The likelihood of a dwarf pair becoming a massive primary with a minor companion is very small, since only XX\% of dwarf pairs have a major tertiary companion. \kc{calculate the number of }

% Alternatively, if the dark matter halo and stellar mass of a dwarf galaxy pair increase enough through cosmic accretion or very minor mergers from $z=4$ to $z=2$, the pair will be reclassified as a massive pair at later times. 

For example, dwarf galaxy merger timescales may be significantly shorter than massive galaxy merger timescales, which would deplete the population of dwarf galaxy pairs by low redshift. 
Additionally, early dwarf galaxy mergers may result in systems that are identified as massive pairs at late times, and massive pairs may grow such that they exceed our mass criteria, thus leading to an apparent overall decrease in the number of pairs. 
The drop in the absolute number of dwarf and massive primaries at low z as shown in Fig.~\ref{fig:counts} indicates that indeed many of these systems leave the mass range specified by our selection criteria between $z=0-1$. 
However, since the number of dwarf primaries is roughly constant between $z=0.5-1.5$, but the number of pairs is still decreasing during this time, this indicates that mergers are occurring during this time, and not yet leaving the mass regime specified. 
In fact, the number of dwarf pairs that are removed from the sample from $z=0$ to $z=2$ is 10 times greater than the total number of massive pairs, indicating that these pairs are not simply accreting matter and moving into the higher mass bin.





\subsection{Comparison to previous pair fraction studies}
We performed a pair fraction analysis for dwarf and massive galaxy pairs in the TNG100 simulation, utilizing the full 6D position and velocity information that simulations enable. 
However this means that direct comparison to pair fractions reported from observations is not straight-forward, as observationally selected pairs uniquely suffer from contamination, due to projected pairs, and restrictive separation criteria, that exclude more widely separated pairs. 
Fortunately, a number of previous studies of pair fractions in the Illustris simulations have used lightcones to create mock images of observationally-motivated fields, from which they select pairs in the same fashion that is typically done with observational data.
These studies select simulated galaxies based on stellar mass, and then use cuts on the projected separation and line of sight velocity difference to select pairs, which permits direct comparison between simulations and observations. 

At the low mass end, \citet{Besla2018} quantified dwarf galaxy pair fractions in Illustris-1, both projected and physical pairs, which enabled a direct comparison of the cosmologically-derived pair fractions with an equivalently selected dwarf pair fraction from SDSS.  
The pair selection criteria for dwarf galaxies, here defined as stellar masses\footnote{The stellar masses of galaxies from the Illustris simulations used in \citet{Besla2018} are from the median of the \citet{Moster2013} abundance matching prescription, the same as used in this work, rather than the stellar mass values from the simulations.} from $2\times10^{8} < M_{*} < 5\times10^{9}\,\Msun$, include a projected separation criteria of $r_{p} < 150\,\kpc$ and a relative line-of-sight velocity difference $\Delta v_{\rm los} < 150\,\kms$. 

By searching for pairs in a projected space while having access to the true 3D positions and velocities of the galaxies, they quantified the contamination fraction and found that up to $\sim40$\% of identified companions were unrelated but appeared to be close due to projection effects. 
In Illustris, they found that the projected major pair fraction of dwarf galaxies at $z\sim0$ is $f_{p,proj.}\sim0.005-0.012$, while the . \kc{where tf are these numbers from??? }
The total pair fraction (including all stellar mass ratios) is $0.032\pm0.005$ for the projected mock catalogs, which are consistent with SDSS equivalently-selected pairs, which find a total pair fraction of $0.035\pm0.3$. 

However, using the full physical catalogs, they find a lower pair fraction of $0.02\pm0.004$. \kc{I *THINK* this is the fraction of TOTAL pairs (N\_1 from the tables)? not just major pairs? otherwise I don't know how it's different.}
For comparison, when we modify our pair selection criteria to match those of \citet{Besla2018}, and require a physical separation $r_{\rm sep}< 150\,\kms$ and a 3D relative velocity $\Delta v_{\rm los} < 150\,\kms$, we find a dwarf major pair fraction of XX at $z=0$, which is slightly SKZ than the \citet{Besla2018} results.
This is/is not surprising because SKZ. 

As the behavior of the major pair fraction as a function of primary stellar mass is consistent between the projected mock catalogs and the SDSS catalogs, they conclude that cosmological simulations can be used to reliably constrain observational pair fractions of dwarf mergers across time. \kc{across time... is this true?} 
Additionally, which projected and physical pair catalogs yield different absolute values of pair fractions, their behavioral trends as a function of mass are nearly identical such that, in terms of trends, projected and physical pair fractions will yield similar results. 


% While the study uses projected separations and line of sight velocities, we can assume that the same \textit{trends} are valid for full-6D selected pairs, such that our pair fraction trends can be reasonably compared to those in observations. 

Separate studies have analyzed the pair fractions of higher mass galaxies in Illustris simulation. 
In particular, \citet{snyder2017} created mock catalogs using lightcones in the Illustris-1 simulation to pick pairs in the same fashion as done in observations. 
Specifically, they consider only major pairs with: $1\times 10^{11}>M_{*,1} >1\times 10^{10.5}Msun$, projected distances between $14\kpc-71\kpc$, and a redshift separation of $\Delta z<0.02(1+z_{pri}$, which corresponds to a velocity separation of $<1.8\time10^4\kms$. 
They find that the major pair fraction for primaries with stellar mass $1\times 10^{10.5-11}Msun$ is constant or decreasing for $z>1$. 
Additionally, they compared to observational studies \kc{what studies} that selected pairs with stellar mass $(M* > 10^{10.8}\Msun)$, $(10 <d/(kpc h-1) < 30)$, and $(Msec/Mpri > 0.1)$, and found good agreement between the measured pair fractions as a function of redshift.
Thus, projected pair fractions in the Illustris simulation are likely representative of those found in observations for massive galaxy pairs.

\citet{snyder2023} extended this work to utilize the TNG simulation, and created mock images of extragalactic survey fields mimicking future planned surveys like JADES.
From the mock images, they calculated the pair fraction for major pairs with projected separations between $5-70 \kpc$ and with redshift separations of $\Delta z< 0.02(1+z)$, and also found flat or decreasing major pair fractions. 
This is in agreement with our work, as our major pair fraction for massive galaxies is approximately constant or decreasing with higher redshift for $z>1$. 
While we cannot directly compare the values of the pair fractions themselves, as the projected and physical pair separations will yield very different pair fractions as shown in \citet{Besla2018}, we can say that the behavioral trends are the same. 
Pair fractions computed using physical separations do not have to account for contamination due to projection effects, which would increase the inferred pair fraction, however our separation criteria is much less stringent, which would also increase the pair fraction. 

% Alejandro's paper connection: studies are finding good agreement between simulations and TNG, as evidenced by this paper. They found that TNG simulated images well approximated the morphologies of galaxies in the KIDS survey, although they were a bit more compact and lumpy. But this means that there is precedent for using TNG in these kinds of comparisons. 

% ? maybe to do: plot with comparison values 
\subsection{Implications for observational pair selection criteria}
    %figures
    %%%%%%%%%%%%%%%%%%%%
    \begin{figure*}[htp]
      \centering
      \includegraphics[width=\textwidth]{sepdist_1000.png}
      \caption{Normalized physical separation distribution for dwarf (top) and massive (bottom) pairs from $z=0$ (left) to $z=4$ (right). Major pairs are shown as solid lines, while minor pairs are dotted. The median separation of each pair type (from Fig.~\ref{fig:sep}) is shown in solid (major) and dotted (minor) vertical black lines. 
      At higher redshift $z=3-4$, a large fraction of both dwarf and massive pairs have low separations ($<$ 125 \& 250 $\kpc$ respectively). However, at low redshift, the separation distribution tends to spread out more evenly across a large range of separations, thus increasing the median.
        }
      \label{fig:sep-dist}
    \end{figure*}
    %%%%%%%%%%%%%%%%%%%%
    %%%%%%%%%%%%%%%%%%%%
    \begin{figure*}[htp]
      \centering
      \includegraphics[width=\textwidth]{frac_selected.png}
      \caption{ The fraction of total pairs selected with additional separation selection criteria as a function of redshift. 
        }
      \label{fig:frac-select}
    \end{figure*}
    %%%%%%%%%%%%%%%%%%%%

    % outline
    \begin{itemize}
        \item examples of typical pair selection criteria (i.e. <75 kpc, velocity crit, etc.)
        \item Show how low separation cuts affect pair fractions      
        \item Sep distribution plot and explanation (text below) 
        \item Plot of \% of total pairs selected as function of z for different separation criteria (or maybe better is to do total as function of separation, for different redshifts?) 
        \item What does this mean for JWST and future surveys?
    \end{itemize}
    
    
% text on the separation distribution plot
Fig.~\ref{fig:sep-dist} shows the full distribution of separations for major and minor dwarf and massive pairs at $z=\{0,1,2,3,4\}$, and includes all 1000 AM realizations. 
For reference, the median at each corresponding redshift from Fig.~\ref{fig:sep} (that is, the median of the medians of each realization) are shown in light black vertical lines.
Each plot is normalized such that the area under an individual curve is 1.
    % the redshift evolution
At $z=4$, all pair types (dwarf and massive, major and minor) have narrow distributions that pile up around 0 $\kpc$\footnote{Note that only pairs with separations $>10\kpc$ are included in this analysis, see Sec.~\ref{sec:methods-pairs}.}, with ranges between $10-150\,\kpc$ for dwarf pairs and $10-300\,\kpc$ for massive pairs. 
The distributions become more and more broad as $z\to0$, where separations range between $\sim10-400\,\kpc$ for dwarf pairs and $\sim10-1000\,\kpc$ for massive pairs.

Since the distribution of separations changes dramatically from $z=4$ to $z=0$, the population of low separation pairs that are selected at $z=4$ includes the majority of true pairs, but this is not the case at low z. 
At low z, separation criteria must be very broad to encompass the majority of pairs. 
Figure~\ref{fig:frac-select} shows the fraction of pairs that have separations less than $\{50, 75, 100, 150\}\kpc$.
At high redshift ($z\sim2-4$) a majority of pairs are included in the selection, however at lower redshift ($z\sim0-1$), the pair completeness decreases substantially, with fewer than XX\% of the total number pairs for XX separation criteria. 

As JWST is expected to image dwarfs (give stellar mass range) out to z=XX in XX survey, the selection criteria for dwarf pairs that are applied will affect the inferred pair fractions greatly. 

%%%%%%%%%%%%%%%%%%%%%%


%%%%%%%%%%%%%%%%%%%%%%%%%%%%%%%%%%%
\pagebreak
\section{Summary and Conclusions}\label{sec:summary}
In this paper, we collect a large set of dwarf and massive subhalo pairs from the Illustris TNG100-1 simulation. 
Pairs are selected from the full volume over all 100 snapshots of the simulation, though the sample size of pairs in the first 20 snapshots ($z>4.2$) is extremely small. 
Major and minor pairs are determined using the stellar mass ratio between the primary and secondary halo using stellar masses from 1000 random realizations of the abundance matching prescription. 
From this pair sample, we calculate the pair fractions and snapshot by snapshot kinematics of 4 different pair types: dwarf major pairs, dwarf minor pairs, massive major pairs, and massive minor pairs.

Our main findings are as follows:
\begin{itemize}
    \item The pair fraction for dwarf and massive pairs does not proceed identically throughout cosmic time. In fact, the two mass scales have opposite behavior at low z. The number of dwarf pairs per dwarf primary decreases from $z=2\to0$, while the number of massive pairs per massive primary is approximately constant from $z=4\to0.25$, with an abrupt increase at lower redshift. 
    \item The median physical separation between the primary and secondary subhalo of a pair increases with decreasing redshift. The typical separation of a dwarf (massive) pair is $120\,\kpc$ ($240\,\kpc$) at $z=0$ and $40\,\kpc$ ($80\,\kpc$) at $z=4$.
    \item The median relative velocity between the primary and secondary subhalo of a pair decreases towards lower redshift. The typical relative velocity of a dwarf (massive) pair is $70\,\kms$ ($175\,\kms$) at $z=0$ and $132\,\kms$ ($290\,\kms$) at $z=4$.
    \item The separation scaled by the total virial radius of the group is similar for dwarf and massive pairs, though massive pairs tend to have lower scaled separations for redshifts between $z=0-2$. The relative velocity scaled by the circular velocity at the virial radius evolves identically for both dwarf and massive pairs, indicating that the relative velocities are self-similar at different mass scales. 
    \item All of the above trends hold true for both major and minor pairs of the same mass range (i.e., the conclusions for the dwarf major pairs hold too for the dwarf minor pairs). 
\end{itemize}

Our results are consistent with prior works that have studied the pair fractions of massive and dwarf galaxies in the Illustris simulations in a projected sense. 
The agreement between the pair fraction trends from projected and from physical pair selection criteria points to the IllustrisTNG simulation's robust ability to reproduce the characteristics of galaxy pairs from observations. 

Additionally, we conclude that:
\begin{itemize}
    \item Dwarf galaxy pairs and massive galaxy pairs likely have different evolutionary and merger processes which result in the opposing behavior of dwarf and massive pair fractions as a function of redshift.
    \item The separation criteria used to observationally identify pairs likely results in an incompleteness of at least XX\% by $z=0$, though is less problematic for high redshift pairs where pairs tend to have smaller separations. 
\end{itemize}

Our conclusions point to the importance of time-evolving pair selection criteria that account for the natural separation increase of pairs at the universe expands. 


\kc{other main takeaways?}

\begin{itemize}
    \item otherMain takeaways: (maybe include these within main findings bullets~)
        \begin{itemize}
            \item lit comp.
            \item dwarfs not equal massives 
            \item JWST imps.
        \end{itemize}
    \item Future work
\end{itemize}

% Findings:
% \begin{itemize}
%     \item 
%     \item Median pair separations increase and relative velocities decrease from high to low redshift. 
% \end{itemize}

%%%%%%%%%%%%%%%%%%%%%%%%
%%%%%%%%%%%%%%%%%%%%%%%%

\bibliography{refs}{}
\bibliographystyle{aasjournal}

\end{document}
