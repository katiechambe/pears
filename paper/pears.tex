\documentclass[twocolumn]{aastex631}


% Packages
\usepackage{microtype}  % ALWAYS!
\usepackage{amsmath}
\usepackage{amsfonts}
\usepackage{amssymb}
\usepackage{multirow}

\definecolor{pink}{RGB}{232,132,161}
\definecolor{yellow}{RGB}{255,213,0}

\newcommand{\kc}[1]{\textcolor{yellow}{\textbf{kc: #1}} }
% \newcommand{\ecite}[1]{\textcolor{pink}{\textbf{: #1}} }
% \newcommand{\e}[1]{\textcolor{yellow}{\textbf{: #1}} }

\newcommand{\remove}[1]{\textcolor{red}{#1}}
\newcommand{\add}[1]{\textcolor{green}{#1}}

\newcommand{\mlg}{\ensuremath{M_{\rm LG}}}
\newcommand{\mmto}{\ensuremath{M_{\rm M31}}}
\newcommand{\mmw}{\ensuremath{M_{\rm MW}}}
\newcommand{\vtan}{\ensuremath{v_\textrm{tan}}}
\newcommand{\vrad}{\ensuremath{v_\textrm{rad}}}
\newcommand{\mud}{\ensuremath{\mu_\delta}}
\newcommand{\mua}{\ensuremath{\mu_\alpha^*}}
\newcommand{\bov}{\ensuremath{\boldsymbol{v}}}
\newcommand{\boldx}{\ensuremath{\boldsymbol{x}}}
\newcommand{\vtrav}{\ensuremath{\bov_{\rm travel}}}
\newcommand{\xtrav}{\ensuremath{\boldx_{\rm travel}}}
\newcommand{\pos}[2]{\ensuremath{\boldx_{\rm #1 \to #2}}}
\newcommand{\vel}[2]{\ensuremath{\bov_{\rm #1 \to #2}}}
\newcommand{\mwbary}{\ensuremath{\textrm{MW}_\textrm{bary}}}
\newcommand{\mwouter}{\ensuremath{\textrm{MW}_\textrm{halo}}}
\newcommand{\mwdisk}{\ensuremath{\textrm{MW}_\textrm{disk}}}
\newcommand{\reflabel}[1]{\ensuremath{^{\mbox{\scriptsize{#1}}}}}

% Style tweaks
% \renewcommand{\twocolumngrid}{\onecolumngrid}
% \setlength{\parindent}{1.1\baselineskip}
% \sloppy\sloppypar\raggedbottom\frenchspacing

%%%%%%%%%%%%%%%%%%%%%%%%%%%%%%%%%%%%%%%%%%%%%%%%%%%%%%%%%%%%%%%%%%%%%%%%%%%%%%%%
\shorttitle{Frequency of galaxy pairs}
\shortauthors{Chamberlain et al.}

%%%%%%%%%%%%%%%%%%%%%%%%%%%%%%%%%%%%%%%%%%%%%%%%%%%%%%%%%%%%%%%%%%%%%%%%%%%%%%%%
\graphicspath{{./}{figures/}}
% Missions
\newcommand{\project}[1]{\textsl{#1}}

% Packages / projects / programming
\newcommand{\package}[1]{\textsl{#1}}
\newcommand{\acronym}[1]{{\small{#1}}}
\newcommand{\github}{\package{GitHub}}
\newcommand{\python}{\package{Python}}
\newcommand{\astropy}{\package{Astropy}}

% Stats / probability
\newcommand{\given}{\,|\,}
\newcommand{\norm}{\mathcal{N}}
\newcommand{\pdf}{\textsl{pdf}}

% Maths
\newcommand{\dd}{\mathrm{d}}
\newcommand{\transpose}[1]{{#1}^{\mathsf{T}}}
\newcommand{\inverse}[1]{{#1}^{-1}}
\newcommand{\argmin}{\operatornamewithlimits{argmin}}
\newcommand{\mean}[1]{\left< #1 \right>}

% Non-scalar variables
\renewcommand{\vec}[1]{\ensuremath{\bs{#1}}}
\newcommand{\mat}[1]{\ensuremath{\mathbf{#1}}}

% Unit shortcuts
\newcommand{\Msun}{\ensuremath{\mathrm{M}_\odot}}
\newcommand{\Mjup}{\ensuremath{\mathrm{M}_{\mathrm{J}}}}
\newcommand{\kms}{\ensuremath{\mathrm{km}~\mathrm{s}^{-1}}}
\newcommand{\pc}{\ensuremath{\mathrm{pc}}}
\newcommand{\kpc}{\ensuremath{\mathrm{kpc}}}
\newcommand{\Mpc}{\ensuremath{\mathrm{Mpc}}}
\newcommand{\kmskpc}{\ensuremath{\mathrm{km}~\mathrm{s}^{-1}~\mathrm{kpc}^{-1}}}
\newcommand{\dayd}{\ensuremath{\mathrm{d}}}
\newcommand{\yr}{\ensuremath{\mathrm{yr}}}
\newcommand{\Myr}{\ensuremath{\mathrm{Myr}}}
\newcommand{\Gyr}{\ensuremath{\mathrm{Gyr}}}
\newcommand{\Kel}{\ensuremath{\mathrm{K}}}
\newcommand{\masyr}{\ensuremath{\mathrm{mas}~\mathrm{yr}^{-1}}}
\newcommand{\muasyr}{\ensuremath{\mu\mathrm{as}~\mathrm{yr}^{-1}}}

% Misc.
\newcommand{\bs}[1]{\boldsymbol{#1}}

% Astronomy
\newcommand{\DM}{{\rm DM}}
\newcommand{\feh}{\ensuremath{{[{\rm Fe}/{\rm H}]}}}
\newcommand{\df}{\acronym{DF}}

% TO DO
\newcommand{\todo}[1]{{\color{red} TODO: #1}}
\newcommand{\apw}[1]{{\color{blue} APW says: #1}}

% Projects
\newcommand{\gaia}{\textsl{Gaia}}
\newcommand{\gaiadr}{\textsl{Gaia}~\acronym{EDR3}}
\newcommand{\hst}{\textsl{HST}}
\newcommand{\ill}{\textsl{Illustris}}
\newcommand{\tng}{\textsl{IllustrisTNG}} 



% Affiliations
\newcommand{\affuofa}{University of Arizona, 933 N. Cherry Ave,
    Tucson, AZ 85721, USA}

%% This is the end of the preamble.  Indicate the beginning of the
%% manuscript itself with \begin{document}.

\begin{document}

\title{
  Frequency of galaxy pairs in the Illustris and Illustris TNG cosmological simulations
}

\author[0000-0001-8765-8670]{Katie~Chamberlain}
\affiliation{\affuofa}

\author[0000-0003-0715-2173]{Gurtina Besla}
\affiliation{\affuofa}


\begin{abstract}
\end{abstract}

%%%%%%%%%%%%%%%%%%%%%%%%%%%%%%%%%%%
\section{Introduction}
\label{sec:intro}


%%%%%%%%%%%%%%%%%%%%%%%%%%%%%%%%%%%
\section{Methods}
\label{sec:methods}
\begin{itemize}
  \item Introduce simulations
  \item outline selection steps
  \item dwarf pairs
  \item massive pairs
  \item include table with parameters used for pairs
  \item plots
    \begin{itemize}
      \item distribution plots
      \item counts plots
    \end{itemize}
\end{itemize}


% selection criterion
\begin{table*}[htb]
  \begin{center}
    \begin{tabular}{lcc}
     & Dwarf Pairs & Massive Pairs \\\hline\hline
    Group Mass & $8\times10^{10} < \rm M_g < 5\times10^{11} M_\odot$ & $1\times10^{12} < \rm M_g < 4\times10^{12} M_\odot$  \\
    % Subhalo Max Mass &  &  \\
    Subhalo Current Mass & $>1\times10^{10} \Msun$ & $> 5\times10^{11} \Msun$ \\
    Primary &$1\times10^{8} < \rm M_* < 5\times10^{9} \Msun$ & $5\times10^{9} < \rm M_* < 1\times10^{11} \Msun$\\\hline
    \end{tabular}
    \caption{\label{table:mass}Selection criteria for dwarf and massive pairs.}
  \end{center}
    \end{table*}

% equivalent snapshot table
\begin{table}[]
  \centering
  \begin{tabular}{l|cc}
    \hline \hline
   & Illustris & TNG  \\ 
  \hline
  z = 0   &   135  &   99   \\
  z = 1   &   85   &   50   \\
  z = 2   &   68   &   33   \\
  z = 3   &   60   &   25   \\
  z = 4   &   54   &   21   \\
  \hline \hline
  \end{tabular}
  \caption{\label{tab:equiv-snapshot} The snapshot number of the original \ill\ and the \tng\ simulations at redshifts $z=0-4$.}
\end{table}



%%%%%%%%%%%%%%%%%%%%%%%%%%%%%%%%%%%
\section{Results}
\label{sec:results}


%%%%%%%%%%%%%%%%%%%%%%%%%%%%%%%%%%%
\section{Discussion}
\label{sec:discussion}


%%%%%%%%%%%%%%%%%%%%%%%%%%%%%%%%%%%
\section{Summary and Conclusions}
\label{sec:summary}

%%%%%%%%%%%%%%%%%%%%%%%%
%%%%%%%%%%%%%%%%%%%%%%%%

\bibliography{refs}{}
\bibliographystyle{aasjournal}

\end{document}
