\documentclass[twocolumn]{aastex631}

% Packages
\usepackage{microtype}  % ALWAYS!
\usepackage{amsmath}
\usepackage{amsfonts}
\usepackage{amssymb}
\usepackage{multirow}

\definecolor{pink}{RGB}{232,132,161}
\definecolor{yellow}{RGB}{255,213,0}

\newcommand{\kc}[1]{\textcolor{yellow}{\textbf{kc: #1}} }
% \newcommand{\ecite}[1]{\textcolor{pink}{\textbf{: #1}} }
% \newcommand{\e}[1]{\textcolor{yellow}{\textbf{: #1}} }

\newcommand{\remove}[1]{\textcolor{red}{#1}}
\newcommand{\add}[1]{\textcolor{green}{#1}}

\newcommand{\mlg}{\ensuremath{M_{\rm LG}}}
\newcommand{\mmto}{\ensuremath{M_{\rm M31}}}
\newcommand{\mmw}{\ensuremath{M_{\rm MW}}}
\newcommand{\vtan}{\ensuremath{v_\textrm{tan}}}
\newcommand{\vrad}{\ensuremath{v_\textrm{rad}}}
\newcommand{\mud}{\ensuremath{\mu_\delta}}
\newcommand{\mua}{\ensuremath{\mu_\alpha^*}}
\newcommand{\bov}{\ensuremath{\boldsymbol{v}}}
\newcommand{\boldx}{\ensuremath{\boldsymbol{x}}}
\newcommand{\vtrav}{\ensuremath{\bov_{\rm travel}}}
\newcommand{\xtrav}{\ensuremath{\boldx_{\rm travel}}}
\newcommand{\pos}[2]{\ensuremath{\boldx_{\rm #1 \to #2}}}
\newcommand{\vel}[2]{\ensuremath{\bov_{\rm #1 \to #2}}}
\newcommand{\mwbary}{\ensuremath{\textrm{MW}_\textrm{bary}}}
\newcommand{\mwouter}{\ensuremath{\textrm{MW}_\textrm{halo}}}
\newcommand{\mwdisk}{\ensuremath{\textrm{MW}_\textrm{disk}}}
\newcommand{\reflabel}[1]{\ensuremath{^{\mbox{\scriptsize{#1}}}}}

% Style tweaks
% \renewcommand{\twocolumngrid}{\onecolumngrid}
% \setlength{\parindent}{1.1\baselineskip}
% \sloppy\sloppypar\raggedbottom\frenchspacing

%%%%%%%%%%%%%%%%%%%%%%%%%%%%%%%%%%%%%%%%%%%%%%%%%%%%%%%%%%%%%%%%%%%%%%%%%%%%%%%%
\shorttitle{Frequency of galaxy pairs}
\shortauthors{Chamberlain et al.}

%%%%%%%%%%%%%%%%%%%%%%%%%%%%%%%%%%%%%%%%%%%%%%%%%%%%%%%%%%%%%%%%%%%%%%%%%%%%%%%%
\graphicspath{{./}{../plots/paper1/}}
\input{math_definitions.tex}

% Affiliations
\newcommand{\affuofa}{University of Arizona, 933 N. Cherry Ave,
    Tucson, AZ 85721, USA}

%% This is the end of the preamble.  Indicate the beginning of the
%% manuscript itself with \begin{document}.

\begin{document}

\title{
  Frequency of dwarf and massive galaxy pairs in the IllustrisTNG cosmological simulation
}

\author[0000-0001-8765-8670]{Katie~Chamberlain}
\affiliation{\affuofa}

\author[0000-0003-0715-2173]{Gurtina Besla}
\affiliation{\affuofa}


\begin{abstract}
  \todo{REMEMBER TO USE 100-1000 REALS IN PLOTS}
  \todo{Include ticks on top and right axis as well}
\end{abstract}

%%%%%%%%%%%%%%%%%%%%%%%%%%%%%%%%%%%
\section{Introduction} \label{sec:intro}


%%%%%%%%%%%%%%%%%%%%%%%%%%%%%%%%%%%
%%%%%%%%%%%%%%%%%%%%%%%%%%%%%%%%%%%
\section{Sample selection}\label{sec:methods}
Our sample data was obtained using a strict set of selection criteria, as outlined in this section.
In Sec.~\ref{sec:methods-sims}, we provide motivation for and details of the simulations utilized in this study.
Sec.~\ref{sec:methods-am} explains the abudance matching prescription used to associate stellar masses with the dark matter subhalos.
Sec.~\ref{sec:methods-crit} provides a detailed outline of our selection criteria and process for picking pairs.
Finally, Sec.~\ref{sec:methods-counts} gives an overview of the properties of our sample.

%%%%%%%%%%%
%%%%%%%%%%%
\subsection{Simulation details}\label{sec:methods-sims}
% what is it that we are trying to study? why use illustris?
To understand the occurance rate and separations of typical galaxy pairs in a cosmological context, we require a simulation that provides a large statistical sample of pairs in both the dwarf and massive galaxy regime. 
Thus, we are limited in our simulation choice to large-volume cosmological simulations. 
However, we also require high resolution simulations such that the dwarf subhalos are resolved well enough to ensure robust statistics at the low mass end, and so we are not limited by our inability to resolve objects down to halo masses of $M_{\rm halo}\approx1\times 10^9\, \Msun$.

% Baryonic processes
Baryonic process are known to affect the dark matter halo structure of low mass ($M_*<1\times10^9\Msun$) subhalos in simulations~\citep[see e.g.][and references therein]{Sales:2022}.
Additionally, the shallow potential well of dwarf galaxies compared to their massive counterparts may lend themselves more easily to disruption from baryonic processes such as supernova feedback, which could change the occurance rate of low mass subhalos in hydrodynamic simulations \kc{check and see if they looked at this in the TNG100 galaxy paper}.
To study the effect of baryonic processes on the rate and separation of pairs, we require cosmological simulations which have both dark matter only runs and to full hydrodynamics.

% details of Illustris and Illustris TNG
The \textit{Illustris Project} ~\citep{vogelsberger14A,nelson15} and \textit{IllustrisTNG Project}~\citep{TNG1, TNG2, TNG3, TNG4, TNG5} are a set of N-body and hydrodynamic cosmological simulations which are perfectly suited for such a study. 
We use the dark matter only and full hydrodynamic runs of both the \ill-1 and \tng100-1 simulations to explore the connection between pairs and baryonic processes. 
Each of these simulations follows the dynamical evolution of $1820^3$ dark matter particles from $z=127$ to $z=0$ through a periodic volume of roughly $100^3$ Mpc$^3$. 
The \ill{} simulations follow a cosmology consistent with \textit{WMAP-9}~\citep{hinshaw13}, while the \tng{} simulations are consistent with \textit{Planck2015}~\citep{Planck2015}.
Specifically, we use the \textit{Illustris-1-Dark}, \textit{Illustris-1}, \textit{TNG100-1-Dark}, and \textit{TNG100-1} simulations (hereafter \illd, \illh, \tngd, and \tngh{} respectively, see Table~\ref{table:sim} for details).

% deets on the hydrodynamics of 
\kc{hydrodynamics details on the difference between the two hydro simulations. also include differences between the two dark simulations, if there are any besides the cosmology. }

% subfind, sublink, and groups and subhalos
We utilize the group catalogs produced by the \texttt{SUBFIND} algorithm ~\citep{springel01,dolag09}. 
These catalogs consist of halos (or "groups"), defined using the Friends-of-Friends (FoF) algorithm~\citep{davis85} to link nearby particles, and their associated subhalos, which are over-dense and gravitationally bound dark matter structures.
We also use the merger trees provided by the \texttt{SUBLINK} algorithm \citep{gomez15}, which tracks subhalos between snapshots using their particle data, in order to trace the mass growth and merger histories of our subhalos.


\begin{table*}[htb]
\begin{tabular}{l|llll}
 & \illd & \illh & \tngd & \tngh\\\hline\hline
 L$_{\rm Box}$ [Mpc] & 106.5 & 106.5 & 110.7 & 110.7 \\
m$_{\rm DM}$ [M$_\odot$] & 7.5$\times10^6$ & 6.3$\times10^6$ & 8.9$\times10^6$ & $7.5\times10^6$\\
m$_{\rm gas}$ & -- & $1.3\times 10^{6}$ & -- & $1.4\times 10^{6}$ \\\hline
\end{tabular}
\caption{\label{table:sim}Parameters of the Illustris simulations used in this analysis.}
\end{table*}



%%%%%%%%%%%
%%%%%%%%%%%
\subsection{Abundance matching prescription}\label{sec:methods-am}
% Why do we need stellar masses?
One of the goals of this study is to determine the impact of baryonic physics on the occurance rate and typical separations of galaxy \kc{subhalo?} pairs in cosmological simulations. 
However, selecting halo pairs by stellar mass ratio using the stellar masses provided by the simulations is not possible in the dark matter only runs. 
Thus, in order ensure the dark matter only and full hydrodynamic samples are self-consistent, we need to associate each of the selected subhalos with a realistic stellar mass.

% What is abundance matching
Abundance matching is one such method of associating a dark matter halo with a stellar mass, assuming a galaxy would form in a halo of that mass. 

However, there is large scatter in the stellar-mass-halo-mass (SMHM) relationship, especially in the dwarf halo regime. 



% How do we utilize it? 
In particular, we use the abundance matching relationship presented in~\citet{moster13}, which provides an analytic recipe to paint stellar masses onto dark matter halos. 
The~\citet{moster13} relationship is a redshift-evolving function of the halo mass, and includes terms to account for the systematic scatter in the SMHM relationship, with larger scatter at lower halo masses.
\kc{why moster instead of another AM relationship? most closely represent data at lower mass? }

To assign stellar masses to each of the dark matter halos, we assume that the stellar mass of both the primary and secondary halo can change as a function of redshift after the secondary has entered the primary's halo and do not immediately cease star formation.
This assumption is consistent with findings from~\cite{Akins2021}, which found that massive dwarf satellites ($M_*\approx 10^8-10^9\, \Msun$) entering MW-mass host halos are rarely quenched, and with~\cite{geha13} which finds that isolated dwarfs ($>1\Mpc$ from a MW-type galaxy) are often star forming and rarely quenched.
Additionally,~\citet{Munshi2021} found that the stellar mass of subhalos at $z=0$ in the XX simulations are more closely correlated with the peak halo mass than the $z=0$ halo mass for halos with peak halo mass $10^8<M_{\rm peak}<10^{11} \Msun$. 

Thus, we treat each of the subhalos as centrals according to the~\citet{moster13} prescription, using the current snapshot redshift and the peak halo mass of the subhalo to calculate the stellar mass. 
This additionally allows us remain robust to scenarios in which a secondary may have formed most of its stars, and then lost a significant portion of its dark matter content through tidal interactions with a primary, but will retain the bulk of its stellar content.

We create 1000 stellar mass realizations from the SMHM relationship, thus accounting for the systematic uncerntainy in subhalo stellar masses. Each realization is treated as an independent sample, such that computed properties (pair frequency, separations, etc) are given as the median each realization. We report the mean of all realizations and error bars indicate the 1$\sigma$ dispersion across the realizations.
%%%%%%%%%%%
%%%%%%%%%%%
\subsection{Selection criteria}\label{sec:methods-crit}
In this section, we outline the selection criteria used to select dwarf and massive subhalo pairs. 
The same selection criteria is utilized for all four simulations (\illd, \illh, \tngd, and \tngh)
In short, blah. 
Detailed steps are as follows. 

\subsubsection{Isolation}
As we are interested in the frequency and separations of galaxy pairs that are undisturbed by nearby massive perturbers, we select subhalo pairs as the most massive subhalos of their group halos. 
We have confirmed that nearly 100\% of the the most massive subhalos in each group are outside of the Hill radius of any nearby subhalo with greater mass, and are thus a rough isolation criteria is inherent in our selection.

\textbf{Definitions:} each of these defines the subhalo at a single snapshot and for a single realization.
\begin{itemize}
  \item Primary - Most massive subhalo in a group by stellar mass
  \item Secondary - 
  \item Major pair - A pair of subhalos consisting of a primary and secondary with stellar mass ratio $1/4 < \rm M_{*,sec}/M_{*, prim} < 1$
  \item Minor pair - A pair of subhalos consisting of a primary and secondary with stellar mass ratio $1/10 < \rm M_{*,sec}/M_{*, prim} < 1/4$
\end{itemize}
Note that the subhalo that is assigned as primary or secondary may change between stellar mass realizations. Also note\kc{that this method may yield pairs where the secondary has a higher subhalo mass than the primary. This one happens ~XX\% of the time.}

\begin{enumerate}
  \item Select groups in some group range that encompasses MCs and MW-M31
  \item For every subhalo in every group with halo mass >1e9Msun, use merger trees to get the maximum masses.
  \item Generate 1000 stellar mass realization using abundance matching so we can account for the spread in the relationship
  \item Generate catalog at each redshift containing: 
  \begin{itemize}
    \item For each realization, and in each group with a single subhalo, add the subhalo and it's properties to a catalog of single halos.
    \item For each realization, and in each group with 2+ halos, find most massive halo by stellar mass, and the second most massive halo by stellar mass, then add to pair catalog. 
    \item If a group has 3+ halos, check the stellar mass of the third most massive object, and see if it is a major/minor companion of the secondary. If not, teritary flag = 0. Otherwise =1.
  \end{itemize}
  \item Select dwarf primaries: for full catalog, most massive halo (in "pairs") or single halos with stellar mass $1\times10^{8} < \msam < 5\times10^{9} \Msun$. 
  \item Select dwarf pairs: for full pair catalog, any pair with stellar mass ratio between 1-1/4 (major pairs) or between 1/4-1/10 (minor pairs). All pairs is the union of these two sets. Any pair or single halo that is outside of these two sets is an "unpaired" halo. Note that unpaired can still have very minor companions. 
\end{enumerate}

The lower mass threshold for subhalos that constitute our dwarf pairs have a halo mass greater than $1\times10^9\Msun$, such that each of our halos are resolved into at least 100 particles in each simulation.  
\kc{why 100?} 

\subsubsection{Primary subhalos}\label{sec:methods-crit-prims}

\subsubsection{Primary subhalos}\label{sec:methods-crit-pairs}

\subsubsection{Stellar masses from the hydro sims}


\subsection{Counts and distributions}\label{sec:methods-counts}

\begin{itemize}
  \item Introduce simulations
  \item outline selection steps
  \item dwarf pairs
  \item massive pairs
  \item include table with parameters used for pairs
  \item plots
    \begin{itemize}
      \item distribution plots
      \item counts plots
    \end{itemize}
\end{itemize}


% selection criterion
\begin{table*}[htb]
  \centering
    \begin{tabular}{lcc}
     & Dwarf Pairs & Massive Pairs \\\hline\hline
    % Group Mass & $8\times10^{10} < \rm M_g < 5\times10^{11} M_\odot$ & $1\times10^{12} < \rm M_g < 4\times10^{12} M_\odot$  \\
    % Subhalo Max Mass &  &  \\
    % Subhalo Current Mass & $>1\times10^{10} \Msun$ & $> 5\times10^{11} \Msun$ \\
    Primary &$1\times10^{8} < \msam < 5\times10^{9} \Msun$ & $5\times10^{9} < \msam < 1\times10^{11} \Msun$\\\hline
    \end{tabular}
    \caption{\label{table:mass}Selection criteria for dwarf and massive pairs.}
    \end{table*}

% equivalent snapshot table
\begin{table*}[]
  \centering
  \begin{tabular}{l|cc|cccc} % increase this to 7 columns
    \hline \hline
   & \multicolumn{2}{c|}{Snapshot number} & \multicolumn{4}{c}{Number of primaries}\\
   & \ill & \tng  & \illd & \illh & \tngd & \tngh \\ 
  \hline
  z = 0   &   135  &   99   & $11431.5_{-24.42}^{68.38}$ & $12650.0_{-25.84}^{46.84}$ & $13255.0_{-32.60}^{56.72}$ & $12035.0_{-69.24}^{20.36}$\\
  z = 1   &   85   &   50   &  $14769.0_{-30.24}^{47.28}$ & $16568.0_{-32.20}^{69.60}$ & $17445.5_{-40.86}^{51.90}$ & $15946.5_{-59.38}^{39.22}$ \\
  z = 2   &   68   &   33   &  $12602.5_{-31.98}^{45.94}$ & $14174.0_{-12.48}^{62.32}$ & $15413.5_{-34.62}^{28.86}$ & $14334.0_{-29.44}^{45.24}$ \\
  z = 3   &   60   &   25   &     $8533.5_{-54.50}^{39.46}$ & $9390.0_{-63.40}^{41.32}$ & $10901.0_{-44.64}^{46.56}$ & $10106.5_{-47.90}^{39.74}$  \\
  % z = 3.7   &   54   &   21   \\
  \hline \hline
  \end{tabular}
  \caption{\label{tab:equiv-snapshot} The snapshot number of the original \ill\ and the \tng\ simulations at redshifts $z=0-3$, as well as the median number dwarf (massive) primaries that were selected with $1\times 10^{8} < \msam < 5\times 10^{9} \rm M_\odot$ ($5\times 10^{9} < \msam < 1\times 10^{11} \rm M_\odot$) in XX abundance matching realizations. \kc{fill in with numbers after doing more realizations} \kc{include takeaway}}
  \end{table*}

  \begin{figure*}[htb]
    \centering
    \includegraphics[width=\textwidth]{counts.png}
    \caption{(Top) Median number of dwarf (left) and massive (right) primaries and pairs per abundance matching realization as a function of redshift.
    Shaded regions show the $1^{\rm st}$ and 99$^{\rm th}$ percentile values of the medians of all realizations. 
    There are approximately 8 times as many dwarf primaries as massive primaries. 
    % dwarf galaxies
    The dwarf primary count (green solid line in top left panel) is at a minimum at $z=4$, rises to a maximum of $\sim15000$ halos per realization by $z=1$, then decreases by $\sim16\%$ to 12000 halos at $z=0$. The dwarf pair count (green dashed line), on the other hand, peaks at $z\sim2$ with 3000 pairs, and decreases to a minimum of about 1000 pairs at $z=0$.
    % Massive galaxies (right)
    The massive primary count (pink solid line in top right panel) behaves similarly to the dwarf primary count, with a minimum count at $z=4$ which rises to a maximum of $\sim2000$ halos per realization at $z\sim1$, then decreases slightly by $\sim5\%$ to $\sim$1800 halos by $z=0$. 
    Unlike dwarf pairs whose pair count peaks at $z=2$, pair count for massive galaxies (pink dashed line) follows similar behavior to the total massive primary count, with a minimum at $z=4$, and increasing to $z=\sim 1$ before leveling off. At very low redshift, the pair count and primary count have opposite behavior, with the pair count \textit{increasing} at very low redshifts of $z=0-0.25$ and \textit{peaking} at $z=0$.
    (Bottom) Fraction of pairs per primary as a function of redshift (i.e., the ratio between dotted line and solid line in each column). Dwarf and massive pair fractions are roughly flat for $z=2.5-4$, but display opposite behavior for $z=0-2.5$. The dwarf pair fraction decreases steadily from $\sim0.24$ to $\sim0.08$, a decrease of roughly $65\%$, while the massive pair fraction is roughly flat between $z=1-2.5$ with an average of $0.31$, before spiking sharply from $\sim 0.29$ to $\sim 0.36$ between $z=0-0.25$, an increase of $25\%$. 
    \todo{(replace with 1000 realizations fig)}
      }
    \label{fig:counts}
  \end{figure*}

  \begin{figure*}[htb]
    \centering
    \includegraphics[width=\textwidth]{stellarmass_distribution.png}
    \caption{\todo{caption, add z=4}
      }
    \label{fig:massratio}
  \end{figure*}



%%%%%%%%%%%%%%%%%%%
%% dwarf figures %%
%%%%%%%%%%%%%%%%%%%


%%%%%%%%%%%%%%%%%%%%%%%%%%%%%%%%%%%

\section{Results}
\subsection{Pair ratios}
Thus, the rate of dwarf pairs and massive pairs \textit{per primary} will evolve distinctly in the dwarf and massive pair case.

\label{sec:results}
\begin{figure*}[htb]
  \centering
  \includegraphics[width=\textwidth]{pairratio.png}
  \caption{(Top) Median fraction (and 1-99 percentile shaded regions) of dwarf (green) or massive (pink) primaries with a major (left) or minor (right) companion. Shaded regions show the $$-$$ percentile range of the median pair fractions of each realization. 
  % 
  (Bottom) The median subtracted difference between the massive pair frction and the dwarf pair fraction, with the $$-$$ percentile range shaded. 
  The fraction of massive primaries with a major or minor companion remains approximately constant for $z>1$, and increases sharply  $z<\sim0.25$ the pair fraction increases sharply to $z=0$. However, the pair fraction for dwarf pairs evolves significantly differently to massive galaxy pairs. The fraction of dwarf primaries with a companion is approximately constant for $z>2$, but decreases monotonically from $z=2$ to $z=0$. \todo{finish caption, replace plot w 1000 reals vers., add 0 horizon} 
    }
  \label{fig:pairratio}
\end{figure*}

\subsection{Kinematic evolution}

\begin{figure*}[htb]
  \centering
  \includegraphics[width=\textwidth]{separation.png}
  \caption{Median physical separation (in kpc) and 1-99 percentile shaded regions of 1000 abundance matching realizations as a function of redshift for dwarf pairs (left, green), and massive pairs (right, pink). Major pairs (high stellar mass ratio) are shown in solid lines, while minor pairs are depicted with dashed lines. Minor dwarf pairs tend to be $\sim10-20\kpc$ closer than major dwarf pairs, while minor massive pairs are between $10-50\kpc$ closer than major pairs. In general, as the redshift \textit{decreases} toward $z=0$, the average physical separation of a dwarf or massive pair \textit{increases}. The evolution of massive pair separations is approximately linear for $z>1$, increasing sharply for $z<0$, while the evolution of the dwarf pairs closely resembles a power law distrbution across the whole redshift range from $z=0-4$. \todo{finish caption, replace with 1000 reals plot}
    }
  \label{fig:sep}
\end{figure*}

\begin{figure*}[htb]
  \centering
  \includegraphics[width=\textwidth]{separation_distribution.png}
  \caption{Normalized physical separation distribution for dwarf and massive pairs (top/green and bottom/pink, respectively) from $z=0$ (left) to $z=3$ (right). Major pairs are shown as solid lines, while minor pairs are dashed. 
  Both dwarf and massive pairs have narrow distributions about low separations at high redshifts, and a wider distribution of separations at lower redshift. 
  \kc{would it be instructive to make this plot, unnormalized, with z=0-3 on same plot to see if low sep pop numbers change drastically}
  Also note than many of the close pairs at high z may merge by $z=0$. \todo{change color of med lines, add z=4, replace with 1000 reals }
    }
  \label{fig:sep-dist}
\end{figure*}





\begin{figure*}[htb]
  \centering
  \includegraphics[width=\textwidth]{velocity.png}
  \caption{Median relative velocity (in \kms) and 1-99 percentile shaded regions of 1000 abundance matching realizations as a function of redshift for dwarf pairs (left, green) and massive pairs (right, pink). Major pairs (high stellar mass ratio) are shown in solid lines, while minor pairs are depicted with dashed lines. Major and minor pairs (in both the dwarf and massive regime) tend to have approximately the same median relative velocity. In addition, as the redshift \textit{decreases} toward $z=0$, the average relative velocity of a pair \textit{decreases} as well. The overall behavior of the redshift evolution is nearly identical for dwarf and massive pairs (though note that the massive pairs have \textit{much} higher relative velocities than the dwarf pairs at all redshifts.) \todo{reaplace with 1000 reals, finish caption, add 0 horizon line}
    }
  \label{fig:vel}
\end{figure*}

\begin{figure*}[htb]
  \centering
  \includegraphics[width=\textwidth]{velocity_distribution.png}
  \caption{
  Normalized velocity distribution for dwarf and massive pairs (top/green and bottom/pink, respectively) from $z=0$ (left) to $z=3$ (right). Major pairs are shown as solid lines, while minor pairs are dashed. 
  The distribution of relative velocities for dwarf and massive pairs move to slightly lower velocities at lower redshifts. The distribution of velocities for major and minor pairs is roughly equivalent. 
  \todo{finish caption, change color of med lines, add z=4, replace with 1000 reals }
    }
  \label{fig:vel-dist}
\end{figure*}





%%%%%%%%%%%%%%%%%%%%%%%%%%%%%%%%%%%
\section{Discussion}
\label{sec:discussion}

Pair ratio fig. -- The vast difference between the dwarf and massive pair fractions suggests \todo{XX - that pairs at different mass scales evolve differently ? Thus, can't just assume a massive pair fraction for a lower mass regime} 

Mean separation fig. -- The increase in low redshift median separations suggests that low separation pairs undergo mergers that remove them from the sample at later snapshots, thus increasing the median separation over time.

Separation dist fig. -- The number of widely separated pairs at low redshift is much greater than at high redshift, so these pairs either form late, or move apart from each other. 

Scaled sep fig. -- \textbf{The difference between the scaled separations of dwarf and massive pairs highlights the mass dependence of the physical distribution of halo pairs, i.e., dwarf pairs are not simply a smaller scale version of massive pairs.}

Scaled vel fig. --   \textbf{Despite the difference between dwarf and massive pairs in physical distribution as indicated by the plot of scaled separations (Figure~\ref{fig:sep-scaled}), the scaled velocity distribution of pairs indicates that dwarf and massive galaxy pairs operate similarly in velocity space. There is little to no mass dependence on the scaled velocity of dwarf or massive pairs.}

\begin{figure*}[htb]
  \centering
  \includegraphics[width=\textwidth]{scaledsep.png}
  \caption{Median scaled separation (separation of a pair divided by the virial radius of its group) as a function of redshift. Dwarf pairs (major (solid) and minor (dashed)) tend to have smaller scaled separations at lower redshifts, leveling out between $z=0-1$. In particular, the secondaries of dwarf major pairs at $z>2$ tend to live more than one virial radius away from the primary. Massive major pairs at low redshift tend to have smaller scaled separations  at $z<\sim2$, but vary significantly at higher redshifts. 
  Minor pairs (right), both dwarf and massive, tend to have smaller separations, tending to live closer than one virial radius between the primary and secondary. Massive minor pairs tend to have smaller scaled separations compared to dwarf pairs at all redshifts.
  \todo{replace with 1000 reals fig, finish caption, add 0 horizon line}
    }
  \label{fig:sep-scaled}
\end{figure*}

\begin{figure*}[htb]
  \centering
  \includegraphics[width=\textwidth]{scaledsep_distribution.png}
  \caption{\todo{finish caption, 1000 reals, change median color, add z=4}}
  \label{fig:sep-scaled-dist}
\end{figure*} 


\begin{figure*}[htb]
  \centering
  \includegraphics[width=\textwidth]{scaledvel.png}
  \caption{Median scaled velocity (relative velocity of a pair divided by the virial velocity), defined as the circular velocity at the virial radius of the group as a function of redshift. 
  Major pairs are on the left (solid lines), minor pairs are on the right in dashed lines.
  All sets of pairs show an increase in scaled velocity at lower redshifts.
  \todo{finish caption, replace with 1000 reals, add 0 horizon line}
  }
  \label{fig:vel-scaled}
\end{figure*} 

\begin{figure*}[htb]
  \centering
  \includegraphics[width=\textwidth]{scaledvel_distribution.png}
  \caption{\todo{finish caption, 1000 reals, change median color, add z=4}}
  \label{fig:vel-scaled-dist}
\end{figure*} 


\kc{add discussion subsection titles}
\subsection{Dwarf and massive pair kinematic differences}
\subsection{Comparison to literature}
\subsection{Implications for observations}


%%%%%%%%%%%%%%%%%%%%%%%%%%%%%%%%%%%
\section{Summary and Conclusions}
\label{sec:summary}

%%%%%%%%%%%%%%%%%%%%%%%%
%%%%%%%%%%%%%%%%%%%%%%%%

\bibliography{refs}{}
\bibliographystyle{aasjournal}

\end{document}
